\documentclass[acmsmall,dvipsnames,nonacm]{acmart}
\settopmatter{printacmref=false, printccs=false, printfolios=true}


\usepackage{listings}
\usepackage[all,cmtip]{xy}

\usepackage{thm-restate}

\definecolor{codegreen}{rgb}{.1,0.5,0.32}
\definecolor{codegray}{rgb}{0.5,0.5,0.5}
\definecolor{codepurple}{rgb}{0.58,0,0.82}
\definecolor{backcolour}{rgb}{0.98,0.98,0.98}
\newcommand\todoR{\textit{todo}}
\newcommand\modified{\textit{modified}}
\newcommand\initialized{\textit{initialized}}
\newcommand\false{\textcolor{magenta}{false}}
\newcommand\true{\textcolor{codegreen}{true}}
\newcommand\skp{\textcolor{codegreen}{skip}}

\newcommand\Circle[1][1.65]{\tikz[baseline=-#1]{\draw(0,0)circle[radius=#1mm];}}
\newcommand{\Square}{\fbox{$\phantom{ a}$}}

\newcommand\bind{\mathbin{\textsf{>>=}}}

\usepackage{array,xspace}
\usepackage[noend]{algpseudocode} %algorithm2e,algorithmic,
\usepackage[shortlabels]{enumitem}
\usepackage{alltt}
\usepackage{relsize}
\usepackage{bbm}
\usepackage{tikz}
\usepackage{threeparttable}
\usepackage{cleveref}
\usepackage{wrapfig}
\usepackage{epigraph}
\usepackage{pgfplots}
\usepackage{subcaption}
\usepackage{multirow}
\usepackage{stmaryrd}
\usepackage{mathpartir}

% Added by DavidD
\usepackage{mathtools}

\usetikzlibrary{automata}
\usetikzlibrary{positioning}
\usetikzlibrary{arrows}
\usetikzlibrary{arrows.meta}
\usetikzlibrary{shapes.misc, positioning}
\usetikzlibrary{shapes}

\tikzset{node distance=3.0cm,
         every state/.style={
         semithick,fill=gray!10},
         initial text={},
         double distance=2pt,
         every edge/.style={
            draw,
            ->,>=stealth',
            auto,
            semithick}}
\tikzset{every state/.append style={inner sep=0.0pt, minimum size=15pt}}
\tikzset{elliptic state/.style={draw,ellipse,semithick,fill=gray!10}}
\tikzset{ematrix/.style={double, double distance=4pt, >={Implies[length=5pt,width=8pt]}}}

\usepackage{todonotes}
\newcommand{\jnf}[1]{\todo[noinline,color=purple]{\footnotesize NF\@: #1}}
\newcommand{\as}[1]{\todo[noinline,color=pink]{\footnotesize AS\@: #1}}
\newcommand{\asinline}[1]{\todo[inline,color=pink,caption={}]{\footnotesize AS\@: #1}}
\newcommand{\mm}[1]{\todo[noinline,color=green]{\footnotesize MM\@: #1}}
\newcommand{\mminline}[1]{\todo[inline,color=pink,caption={}]{\footnotesize MM\@: #1}}

\usepackage[createShortEnv]{proof-at-the-end}

\makeatletter
\def\@copyrightspace{\relax}
\makeatother

% do not indent theorems, proofs, definitions, etc.
\makeatletter
\def\@acmplainindent{0pt}
\def\@acmdefinitionindent{0pt}
\def\@proofindent{\noindent}
\makeatother

\setcopyright{none}
\copyrightyear{2023}
\acmYear{2023}
\acmDOI{}
\acmConference[]{}{}{}

%% Macros %%
\newtheorem{remark}{Remark}
\newtheorem{question}{Q}

%%%%%%
\setcitestyle{numbers}

%%\newcommand\cont{\delta{_{\alpha\beta}}}
%%\newcommand\obs{\varepsilon{_{\alpha\beta}}}
\newcommand\At{\mathsf{At}}
\newcommand\Pk{\mathsf{Pk}}
\newcommand\sympk{\widehat\alpha}
\newcommand\sympkp{\widehat\beta}
\newcommand\symR{\widehat{R}}
\newcommand\pk{\alpha}
\newcommand\pkp{\beta}

\newcommand\dup{\mathsf{dup}}
\newcommand\bisim{\mathsf{bisim}}
\newcommand\Exp{\mathsf{Exp}}
\newcommand\sw{\mathsf{sw}}
\newcommand\pt{\mathsf{pt}}
\newcommand\leaves{\mathsf{leaves}}
\newcommand\symTrans{\mathsf{trans}}
\newcommand\binopFDD{\mathsf{binopFDD}}
\newcommand\ddd{\cdot\dup\cdot}
\newcommand\GSdef{\Pk \cdot (\Pk\cdot\dup)^\star\cdot\Pk}
\newcommand\GSdefold{\At \cdot (\Pk\cdot\dup)^\star\cdot\At}
\newcommand\GS{\mathsf{GS}}
\newcommand\accept{\mathsf{accept}}
\newcommand\NetKAT{\textsf{NetKAT}\xspace}
\newcommand\Pred{\textsf{Pred}\xspace}

\newcommand\BDD[2]{\textsf{BDD}_{#1}[#2]}
\newcommand\FDD{\textsf{FDD}}
\newcommand{\fddn}[4]{#1 \stackrel ? = \overline{\{ #2\mapsto #3\}} \diamond #4}

\newcommand\btodo{\fbox{to do.}}

\newcommand{\Sem}[1]{\llbracket #1 \rrbracket}
\newcommand{\pSem}[1]{[ #1 ]}
\newcommand{\fddSem}[1]{\llbracket #1 \rrbracket}
\newcommand{\val}[1]{(\!| #1 |\!)}
\newcommand{\approxp}[1]{\underset{#1}\approx}
\newcommand{\bubble}[1]{\node[state,minimum size=8pt] {#1};}
\newcommand{\bubbleinline}[1]{\begin{tikzpicture}\bubble{#1}\end{tikzpicture}}

\newcommand\pow[1]{\mathcal{P}(#1)}

\newcommand\zero{\bot}%\mathsf{drop}}
\newcommand\one{\top}%\mathsf{skip}}


\newcommand\Fwd{\textsc{Forward-Step}}
\newcommand\Bkwd{\textsc{Backward-Step}}
\newcommand\BkwdWP{\textsc{Backward-WP}}
\newcommand\FwdSing{\textsc{Forward-Single}}
\newcommand\BkwdSing{\textsc{Backward-Single}}
\newcommand\FwdSym{\textsc{Forward-Symbolic}}
\newcommand\BkwdSym{\textsc{Backward-Symbolic}}
\newcommand\A{\mathcal{A}}
\newcommand\B{\mathcal{B}}
\newcommand\epsA{\epsilon^\A_\pk}
\newcommand\epsB{\epsilon^\B_\pk}
\newcommand\delA{\delta^\A_{\pk\pkp}}
\newcommand\delB{\delta^\B_{\pk\pkp}}

\newcommand\farr{\underset{\textsc{Fwd}}{\leadsto}}
\newcommand\barr{\underset{\textsc{Bwd}}{\leadsto}}
\newcommand\farrSing{\underset{\widehat{\textsc{Fwd}}}{\rightsquigarrow}}
\newcommand\barrSing{\underset{\widehat{\textsc{Bwd}}}{\rightsquigarrow}}
\newcommand\farrSym{\underset{\widehat{\textsc{Fwd}}}{\leadsto}}
\newcommand\barrSym{\underset{\widehat{\textsc{Bwd}}}{\leadsto}}
\newcommand\barrWP{\underset{\textsc{WP}}{\leadsto}}

\newcommand\visited{\textsf{visited}}
\newcommand\wlG{\textsf{wlG}}
\newcommand\wlB{\textsf{wlB}}

\newcommand\gclbox{\quad[\!]\quad}
%%%



\title{Material Deleted from Main Draft}

\begin{document}

\maketitle

Historically, methods for computing bisimulations of automata \cite{leapfrog, bonchi-pous2013} have been based
on either on Moore's algorithm for minimization \cite{moore} or Hopcroft and
Karp's algorithm \cite{hopcroft-karp1971}. In this section, we develop a technique
that generalizes the two approaches to a unified generic scheme for \NetKAT automata.
The idea is to proceed by building up simultaneously a partial bisimulation and
partial proof of non-existence of a bisimulation. On the one hand, we might
finish building the bisimulation, or exhaust possible steps for disproving
(proving equivalence of $\A$ and $\B$). In either case, this will be enough to
conclude the automata are bisimilar. On the other hand, we may derive a conflict
(proving inequivalence).

To carry out this approach, we compute two sets of triples from $\Pk\times
S\times T$, which we call $G$ (for ``Good'' triples) and $B$ (for ``Bad''
triples). The triples in $G$ are needed for every possible
bisimulation and triples in $B$ can never be in any bisimulation. To initialize
the good set $G$, we start by looking at condition (a)
of the definition of bisimulation, and we set
\[G = \Pk\times\{s_0\}\times\{t_0\}.\]
Similarly, by looking at
condition (b)(i) of the definition, initialize
\[B = \{(\pk, s, t) \in \Pk\times S\times T \mid \epsA(s) \neq \epsB(t)\}.\]

At this point, we proceed by applying the following rewrite rules for $B$ and
$G$, in any order, adding a triple to one set or the other:

%% Inference rules:
\begin{mathpar}
\frac {(\pk, s, t)\in G \qquad \pkp\in \Pk}
      { G \farr G \uplus \{(\pkp, \delA(s), \delB(t))\}}
      \ \Fwd\\

\frac {(\pkp, s', t')\in B \qquad \pk\in \Pk \qquad \delA(s) = s'\qquad \delB(t) = t'}
      {B \barr B \uplus \{(\pk, s, t)\}}
      \ \Bkwd
\end{mathpar}

\begin{figure}
\begin{algorithmic}
  \State $G \gets \Pk\times\{s_0\}\times\{t_0\}$;
  \State $B \gets \{(\pk, s, t) \in \Pk\times S\times T \mid \epsA(s) \neq \epsB(t)\}$;
  \While{$G \cap B = \varnothing \wedge (\exists G'\colon G\farr G') \wedge (\exists B'\colon B \barr B')$}
    \State{$G \farr G'; G \gets G' \gclbox B \barr B'; B \gets B'$}
  \EndWhile
  \Return $G\cap B = \varnothing$
\end{algorithmic}
  \caption{Pseudocode for unified bisimulation algorithm schema. The $[\!]$
  inside the loop stands for guarded nondeterministic choice between the two
  possible steps.} \label{fig:unified-alg}
\end{figure}

The algorithm schema arising from these inference rules is shown in
\Cref{fig:unified-alg}.
Note that the triples added to $G$ are those which would be included in the
bisimulation computed by the naive algorithm in \Cref{fig:naive}.
While applying rules, if we ever discover that $B \cap G \neq
\varnothing$, then we can stop and report that no bisimulation is possible and
hence the automata are not language equivalent.  The conjunction in
the while loop condition may be somewhat surprising at first. As soon as
\emph{one} of the inference rules can no longer be applied, then we have enough
to conclude the automata are bisimilar.  We justify these claims in the next section.

\subsection{Correctness of the unified framework}\label{sec:correct-unified}

We are ready to show the correctness of \Cref{fig:unified-alg} as a decision
procedure for the equivalence of two deterministic \NetKAT
automata.

\begin{lemma}\label{lem:bcapg}
At any point during the algorithm in \Cref{fig:unified-alg}, if $B \cap G \neq \varnothing$ then no bisimulation
exists between $\mathcal{A}$ and $\mathcal{B}$.
\end{lemma}
\begin{proof}
  Suppose $\A$ and $\B$ are language equivalent, and let $R$ be a bisimulation
  between $\A$ and $\B$. We start by proving two invariants about the main
  loop which hold in this case, and we will conclude by finding that $B \cap G
  \neq \varnothing$ contradicts the existence of $R$.
  \begin{enumerate}
    % Invariant 1
    \item We will show that
      \begin{equation}
        \tag{I1}\label{eqn:i1}
        G \subseteq R
      \end{equation}
      is an invariant of the loop (i.e., \ref{eqn:i1} holds each time the loop guard is checked).
      \begin{itemize}
        \item \emph{Initialization.} We have that \ref{eqn:i1} holds the first
          time reaching the top of the loop. Before the loop, $G$ is
          initialized with triples of the form $(\pk, s_0, t_0)$, for each
          $\pk\in\Pk$. Each such triple is in $R$ by requirement (a) of the
          definition of bisimulation.
        \item \emph{Maintenance.} \Cref{eqn:i1} is maintained by the loop body.
          Assume \ref{eqn:i1} and the loop guard hold. If we modify $B$ by
          applying \Bkwd, then $G\subseteq R$, trivially, by \ref{eqn:i1}. So assume we
          modify $G$ by applying \Fwd. Applying \Fwd says that we add a triple
          $(\pkp,\delA(s),\delB(t))$ corresponding to some $(\pk,s,t)\in G$
          already. By $\ref{eqn:i1}$, $(\pk,s,t,)\in R$. By (b)(ii) of the
          definition, this means $(\pkp,\delA(s),\delB(t)) \in R$. Thus after
          adding $(\pkp, \delA(s), \delB(t))$ to $G$, we still have that
          $G\subseteq R$.
      \end{itemize}

    % Invariant 2
    \item We now show that
      \begin{equation}
        \tag{I2}\label{eqn:i2}
        B \cap R = \varnothing
      \end{equation}
      is an invariant of the loop.
      \begin{itemize}
        \item \emph{Initialization.} \Cref{eqn:i2} holds the first time reaching the top of the loop.
          Before the loop, $B$ is initialized with all the triples $(\pk, s,t)$ for which
          $\epsA(s) \neq \epsB(t)$. None of these triples can be in $R$ because of
          requirement (b)(i) of the definition.
        \item \emph{Maintenance.} \Cref{eqn:i2} is maintained by the loop body.
          Assume \ref{eqn:i2} and the loop guard hold. If we apply \Fwd,
          modifying $G$, then $B\cap R =\varnothing$, trivially, by
          \ref{eqn:i2}. So assume we modify $B$ by applying \Bkwd. Applying
          \Bkwd means adding a triple $(\pk,s,t)$ to $B$ corresponding to a
          triple $(\pkp, s', t')\in B$ already, with $\delA(s)=s'$ and
          $\delB(t)=t'$. By \ref{eqn:i2}, $(\pkp, s',t')\notin R$. This means
          that $(\pk, s, t)$ cannot be in $R$, since if it were, requirement
          (b)(ii) of the definition and would mean that $(\pkp, s', t')$ would
          need to be in $R$ also. So we add $(\pk, s,t)$ to $B$, and since
          $(\pk,s,t)\notin R$, we have $B\cap R = \varnothing$.
      \end{itemize}
  \end{enumerate}
  Now suppose at some point during execution that $B \cap G \neq \varnothing$.
  Without loss of generality, we are at the top of the loop (including the
  possibility that the guard is about to fail), because the loop only executes
  a single instruction with effect and nothing follows the loop except the
  return statement. So we know that both \ref{eqn:i1} and \ref{eqn:i2} hold.
  Moreover, $G \subseteq R$ (by \ref{eqn:i1}), combined with $B\cap G \neq
  \varnothing$ means that $B \cap R \neq \varnothing$, which contradicts
  \ref{eqn:i2}. We conclude there can be no such $R$.
\end{proof}

\begin{lemma}\label{lem:termination}
The algorithm in \Cref{fig:unified-alg} terminates.
\end{lemma}
\begin{proof}
Because of the disjoint union in both inference rules, on every iteration of the
loop we either add a triple to $G$ or add a triple to $B$. Thus, the value
of $n = |\Pk\times S\times T - G - B|$ is finite initially and decreases every
loop iteration. Clearly when $n = 0$, it will either be the case that $G \cap B
\neq \varnothing$, or one of the rules will be exhausted. In either case the
loop condition fails and we terminate.
\end{proof}

\begin{theorem}\label{thm:unified-correct}
The algorithm in \Cref{fig:unified-alg} returns $\true$ if there is a
bisimulation between $\mathcal{A}$ and $\mathcal{B}$ and returns $\false$ if
there is no such bisimulation.
\end{theorem}
\begin{proof}
  By \Cref{lem:termination}, we reach the return statement on the last line.

  If we return $\false$, this means that $B \cap G = \varnothing$. Thus, by \Cref{lem:bcapg}, no bisimulation
  exists for $\A$ and $\B$.

  So assume we return $\true$. We need to show a bisimulation exists.
  In that case we know the loop guard is false: i.e. either that $G\cap B \neq \varnothing$,
  that $G'\colon G\farr G'$, or that $\nexists B'\colon B \barr B'$.

  We consider these in cases:
  \begin{enumerate}
    \item $G\cap B \neq \varnothing$. This is impossible because we assumed we
      return $\true$.
    \item $\nexists G'\colon G \farr G'$, i.e. $\Fwd$ cannot be applied. In
    this case, $G$ is a bisimulation:
        \begin{itemize}
          \item The triples required by (a) of the definition are added to $G$
            in the initialization step.
          \item All triples for which (b)(i) does not hold are added to $B$
            in initialization. Thus, from $B \cap G = \varnothing$, we conclude that
            (b)(i) holds for all triples in  $G$.
          \item That item (b)(ii) holds follows immediately from the fact
          that $\nexists G'\colon G \farr G'$.
        \end{itemize}
    \item $\nexists B'\colon B \barr B'$, i.e. $\Bkwd$ cannot be applied. In
    this case, $B^C$ is a bisimulation:
        \begin{itemize}
          \item The triples required by (a) of the definition are added to $G$
          in initialization, so we conclude from the fact that $B \cap G = \varnothing$
          that they are \emph{not} in $B$.
          \item All triples for which (b)(i) does not hold are added to $B$
          initially, so (b)(i) holds for all triples not in $B$.
          \item That item (b)(ii) holds for the triples \emph{not} in $B$ follows
          from the fact that $\nexists B'\colon B \barr B'$.
        \end{itemize}
      \end{enumerate}
\end{proof}

\section{Toward a more practical algorithm}

The unified algorithm in \Cref{fig:unified-alg} is correct,
but it is somewhat unsatisfying to try to implement. In particular:
\begin{itemize}
\item There are many triples to check to even decide whether a rule is
applicable.
\item Even if a rule is applicable, there may be many applicable triples to
choose from. Which one do we take?
\end{itemize}

In \Cref{sec:sympackets} and \Cref{sec:visitedset}, we develop improvements to
the algorithm of \Cref{fig:unified-alg} that help us to answer the first
question. We address the second question in \Cref{sec:strategies}.

\subsection{Symbolic packets}\label{sec:sympackets}

The algorithm in \Cref{fig:unified-alg} repeats unnecessary work by applying
rules to individual triples in $\Pk\times S\times T$. We can be much more
efficient by applying the symbolic representation introduced in
\Cref{sec:sym-automata}.
In particular, the symbolic representation allows us to apply the inference rule
at once for any set of packets that satisty the same premises.

In this approach, our sets $G$ and $B$ have the same type as the symbolic
bisimulation of \Cref{sec:sym-automata}. That is, for each $s\in S$ and each $t\in T$, we
define:
\begin{mathpar}
G_{s,t}\colon \mathcal{P}(\Pk) \\
B_{s,t}\colon \mathcal{P}(\Pk)
\end{mathpar}

For these symbolic sets, we define $G \cap B$ by:
\[G \cap B \triangleq \bigcup_{s \in S, t\in T} G_{s,t} \cap B_{s,t}\]

In addition, we adapt the inference rules to the symbolic setting as follows:
\begin{mathpar}
\frac {\sympkp = \{\pkp\in\Pk \mid \exists \pk\in G_{s,t}\colon \delA(s)=s' 
        \wedge \delB(t) = t'\}}
      {G_{s,t} \farrSym G_{s',t'} \uplus \sympkp}\ \FwdSym \\

\frac {\sympk = \{\pk\in\Pk \mid \exists \pkp\in B_{s,t}\colon \delA(s)=s'
        \wedge \delB(t) = t'\}}
      {B_{s,t} \barrSym B_{s',t'} \uplus \sympk}\ \BkwdSym
\end{mathpar}

Note that the $\uplus$ in the above rules is disjoint union on symbolic
packets. The rules say that, for a particular pair of states $(s,t)$
transitioning to a pair of states $(s',t')$, in a single step we can consider
all the concrete packets $(\pk,\pkp)$ that transition from $s$ to $s'$ in
$\mathcal A$ and from $t$ to $t'$ in $\mathcal B$. The trick is to collect all
such packets into symbolic packets.

The resulting algorithm is shown in \Cref{fig:sym-alg}.

\begin{figure}
\begin{algorithmic}
  \State $G_{s_0,t_0} \gets \Pk$;
  \For{$(s,t) \in S\times T$}
    \State $B_{s,t} \gets \{ \pk \mid \epsA(s) \neq \epsB(t) \}$;
  \EndFor
  \While{$G\cap B\neq \varnothing \wedge
         (\exists s,t\colon G_{s,t} \farrSym G'_{s,t}) \wedge
         (\exists s,t\colon B_{s,t} \barrSym B'_{s,t})$}
    \State Choose $s,t\colon G_{s,t} \farrSym G'_{s,t}; G_{s,t} \gets G'_{s,t}$;
    \State $\gclbox$
    \State Choose $s,t\colon B_{s,t} \barrSym B'_{s,t}; B_{s,t} \gets B'_{s,t}$;
  \EndWhile
  \Return $B\cap B = \varnothing$;
\end{algorithmic}
\caption{Pseudocode for bisimulation algorithm using symbolic packets. Each
place that a set of packets appears in the code, we use a symbolic
representation} \label{fig:sym-alg}
\end{figure}

\subsection{Visited set}\label{sec:visitedset}
The improvement of using symbolic packets allows us to collapse many steps of
the bisimulation into single steps, but it does not yet tell us which step to
take next---the algorithm still just says to ``find an applicable inference rule''.
Here, we build on the symbolic packets by also organizing the search using a
``visited set'' to keep track of triples we have already investigated.

In order to present the visited-set algorithm more cleanly, we introduce a
notational convenience to allow us to apply the inference rule to a specific
rule in $G$ or $B$:
\begin{mathpar}
\frac {(\pk,s,t)\in G}
      {(\pk,s,t) \farrSing \{(\pkp, s', t') \mid \delA(s) = s' \wedge \delB(t) = t'\}}
      \ \FwdSing\\
\frac {(\pkp, s', t') \in B}
      {(\pkp, s', t') \barrSing \{(\pk, s, t) \mid \delA(s) = s' \wedge \delB(t) = t'\}}
      \ \BkwdSing
\end{mathpar}

The visited-set form of the symbolic packet algorithm is shown in \Cref{fig:vs-alg}.
We can equivalently view this formulation as a ``worklist'' algorithm, where the
set $G \cup B - \visited$ serves as the worklist.

\begin{figure}
\begin{algorithmic}
  \State $G[s_0][t_0] \gets \Pk$;
  \For{$(s,t) \in S\times T$}
    \State $B_{s,t} \gets \{ \pk \mid \epsA(s) \neq \epsB(t) \}$;
    \State $\visited_{s,t} \gets \varnothing$
  \EndFor
  \While{$G\cap B\neq \varnothing \wedge G \cup B - \visited \neq \varnothing$}
    \State $(\sympk, s, t) \gets \text{choose}(G \cup B - \visited)$;
    \State $\visited_{s,t} \uplus \sympk$;
    \State $(\sympk, s, t) \farrSing G'; G \gets G'$;
    \State $\gclbox$
    \State $(\sympk, s, t) \barrSing B'; B \gets B'$;
  \EndWhile
  \Return $G\cap B = \varnothing$;
\end{algorithmic}
\caption{Pseudocode for bisimulation using both symbolic packets and worklists.}
\label{fig:vs-alg}
\end{figure}


\subsection{Forward and Backward Projections} \label{sec:strategies}

We first remark that for each iteration on the algorithm of
\Cref{fig:unified-alg}, we have maintained the free choice of whether to apply
the rules in the forward or backward direction in the inner loop. Since we can
terminate by exhausting \emph{either} rule, we can just as well select one or
the other rule at the beginning, and apply only that one to exhaustion.

We illustrate these two projections using two examples.
\begin{example} Consider the expression, $e_1$, from before, along side the
following expression, $e_2$:
\begin{align*}
e_1 &\triangleq
     x=1 + y=2\cdot y\gets7\cdot\dup\cdot(z=1\cdot y\gets3 +  z=2\cdot
     z\gets4\cdot\dup\cdot y\gets1)\\
e_2 &\triangleq x=1 + y=2\cdot 
     ((y=2\cdot (z=1 + z=2)\cdot y\gets 7 + y=7\cdot z=2\cdot z\gets 4)\cdot
     \dup)^\star \\
    &\cdot (y=7\cdot z=4\cdot y\gets 1 + z=1\cdot y=7\cdot y\gets 3)
\end{align*}

\paragraph*{Forward-only}
We show the computation of the bisimulation between $e_1$ and $e_2$ in
\Cref{fig:ex-fwd} using only the \FwdSym\ rule. We begin on the left side by
initializing $G$ with only $G_{s0,t0} = \Pk$.  (marked \bubbleinline{1}). First,
since $B_{s0,t0} = \varnothing$, we have $G_{s0,t0} \cap B_{s0,t0} =
\varnothing$. From here, we consider the transitions that are available on each
automaton for different symbolic packets.
We find that $\delA(s0) = s1$ and $\delB(t0) = t1$ for
\begin{mathpar}
\sympk = \Pk\qquad\sympkp = \{ \pkp \mid \pkp.y = 7 \wedge ( \pkp.z = 1 \vee
\pkp.z = 2)\}\\
\end{mathpar}
So we continue from $G_{s1,t1} = \sympkp$, shown at \bubbleinline{2}
\footnote{Note that transitions are possible to
drop states for other packets; these are checked by the algorithm but omitted in
the illustration.}. For the pair $(s1, t1)$, we check that all the packets within
this set have the same behavior on the observation functions (i.e., that $G_{s1,
t1} \cap B_{s1,t1} = \varnothing$).
Indeed, we see that of those in the current packet set, only those with
$z=1$ are accepted, and they get the same update ($y\gets 3$). We proceed to
check available transitions, bringing us to the set of packets at
\bubbleinline{3}. Again, the observation function behavior is the same for
packets in the set (with update $y\gets 1$). The state $t1$ has a self loop, but
it is not available to the packets in the set, meaning everything else is
dropped. We stop and conclude the automata are bisimilar.


\begin{figure}
\begin{center}
\begin{tikzpicture}
% A_1 for e_1
\node[state, initial] (s0) {s0};
\node[state, right of=s0, node distance=1.5in] (s1) {s1};
\node[state, right of=s1, node distance=1.5in] (s2) {s2};
\node[draw=none, below of=s0, node distance=27pt] {$x=1$};
\node[draw=none, below of=s1, node distance=27pt] {$z=1\cdot y\gets 3$};
\node[draw=none, below of=s2, node distance=27pt] {$y\gets 1$};
\draw (s0) edge node{$y=2\cdot y\gets 7$} (s1);
\draw (s0) edge[ematrix] ++ (0,-0.75);
\draw (s1) edge[ematrix] ++ (0,-0.75);
\draw (s1) edge node{$z=2\cdot z\gets 4$}  (s2);
\draw (s2) edge[ematrix] ++ (0,-0.75);

% A_2 for e_2
\node[state,initial, below of=s0, node distance=1.75in] (t0) {t0};
\node[state, right of=t0, node distance=2in] (t1) {t1};
\node[draw=none, below of=t0, node distance=27pt] {$x=1$};
\node[draw=none, below of=t1, node distance=30pt, align=left] {$y=7\cdot z=4\cdot y\gets 1 +$\\
                                                               $z=1\cdot y=7\cdot y\gets 3$};
\draw (t0) edge[ematrix] ++ (0,-0.75);
\draw (t1) edge[ematrix] ++ (0,-0.75);
\draw (t0) edge node {$y=2\cdot(z=1 + z=2)\cdot y\gets 7$} (t1);
\draw (t1) edge[loop right] node [right, align=left]{$y=2\cdot(z=1 + z=2)\cdot y\gets 7 +$\\
                                                     $y=7\cdot z=2\cdot z\gets 4$} (t1);

% Bisimulation relation
\draw[style=dashed] (s0) -- node[left] (pk) {\fbox{$\varnothing$}} (t0);
\draw[style=dashed] (s1) -- node[left] (y7) {\fbox{\parbox{60pt}{\centering
$y\neq 7 \vee z=2)$}}} (t1);
\draw[style=dashed] (s2) -- node[right] (z4y7) {\fbox{$z\neq4 \vee y\neq7$}} (t1);

% Bisim steps
\node[state, minimum size=8pt, left of=pk, node distance=16pt] (b1) {1};
\node[state, minimum size=8pt, left of=y7, node distance=40pt] (b2) {2};
\node[state, minimum size=8pt, left of=z4y7, node distance=38pt] (b3) {3};

\end{tikzpicture}
\end{center}
\caption{Example run of the algorithm in \Cref{fig:vs-alg}, using only the
forward step rule. The boxes show the disallowed triples $B$ as they are
identified.}\label{fig:ex-fwd}
\end{figure}
\end{example}

\paragraph*{Backward-only}
We now work the same example again (in \Cref{fig:ex-bwd}), this time using
only the \BkwdSym\ inference rule. At \bubbleinline{1}, we compute $B_{s2,t1}$.
Since the update
$y\gets 1$ applies to all packets at $s2$, we see that the set we are looking
for is those have this behavior at $t1$, hence we have $B_{s2,t1} = z\neq4 \vee
y\neq7$. On the next iteration, this triggers a look at
$B_{s1,t1}$, since we have $\delA(s1) = s2$ and $\delB(t1) = t1$ for the
symbolic packets
\begin{mathpar}
\sympk = \{\pk \mid \pk.z=2 \wedge (\pk.y=2 \wedge \pk.y=7) \}\qquad
\sympkp =  \{\pkp \mid \pk.z=4 \wedge \pk.y=7 \}
\end{mathpar}

At \bubbleinline{2}, we see the packets satisfying $ $ have different output
packets, and we update $B_{s1,t1}$ accordingly.  Proceeding backwards along
transitions to \bubbleinline{1}, we find that the guards are such that, at
$(s0,t0)$, all packets will either be dropped by both automata or proceed to
$(s1, t1)$ \emph{but avoid landig in $B_{s1,t1}$, at \bubbleinline{2}}. We
conclude that $(s0,t0)$ have the same behavior for all packets and thus that the
automata are bisimilar.  The drawing omits the computation of $B_{s1,t0},
B_{s2,t0}$, and $B_{s0,t1}$ since they do not yield any backward steps.

\begin{figure}
\begin{tikzpicture}
\node[state, initial] (s0) {s0};
\node[state, right of=s0, node distance=1.5in] (s1) {s1};
\node[state, right of=s1, node distance=1.5in] (s2) {s2};
\node[draw=none, below of=s0, node distance=27pt] {$x=1$};
\node[draw=none, below of=s1, node distance=27pt] {$z=1\cdot y\gets 3$};
\node[draw=none, below of=s2, node distance=27pt] {$y\gets 1$};
\draw (s0) edge node{$y=2\cdot y\gets 7$} (s1);
\draw (s0) edge[ematrix] ++ (0,-0.75);
\draw (s1) edge[ematrix] ++ (0,-0.75);
\draw (s1) edge node{$z=2\cdot z\gets 4$}  (s2);
\draw (s2) edge[ematrix] ++ (0,-0.75);

\node[state,initial, below of=s0, node distance=1.75in] (t0) {t0};
\node[state, right of=t0, node distance=2in] (t1) {t1};
\node[draw=none, below of=t0, node distance=27pt] {$x=1$};
\node[draw=none, below of=t1, node distance=30pt, align=left] {$y=7\cdot z=4\cdot y\gets 1 +$\\
                                                               $z=1\cdot y=7\cdot y\gets 3$};
\draw (t0) edge[ematrix] ++ (0,-0.75);
\draw (t1) edge[ematrix] ++ (0,-0.75);
\draw (t0) edge node {$y=2\cdot(z=1 + z=2)\cdot y\gets 7$} (t1);
\draw (t1) edge[loop right] node [right, align=left]{$y=2\cdot(z=1 + z=2)\cdot y\gets 7 +$\\
                                                     $y=7\cdot z=2\cdot z\gets 4$} (t1);

% Bisimulation relation
\draw[style=dashed] (s0) -- node[left] (l) {\fbox{$\Pk$}} (t0);
\draw[style=dashed] (s1) -- node[left] (c) {\fbox{$y=7\wedge z\neq 4$}} (t1);
\draw[style=dashed] (s2) -- node[right] (r) {\fbox{$y=7 \wedge z=4$}} (t1);

% Bisim steps
\node[state, minimum size=8pt, left of=l, node distance=16pt] (b3) {3};
\node[state, minimum size=8pt, left of=c, node distance=38pt] (b2) {2};
\node[state, minimum size=8pt, right of=r, node distance=38pt] (b1) {1};
\end{tikzpicture}
\caption{Example run of bisimulation procedure \Cref{fig:vs-alg} taking
\BkwdSym\ steps only. The boxes show $B$ as it is built. Some triples in $B$
which do not have backward steps are elided.}
\label{fig:ex-bwd}
\end{figure}

\section{Reverse of \NetKAT Expressions and Automata}

Classic theorems of formal language theory state that one can reverse a regular
expression by recursively inverting the concatenations in its tree structure.
Similarly, a nondeterministic automaton (NFA) can be reversed by interchanging the
start and final states and reversing the direction of all of the transitions
(Brzozowksi's algorithm for DFA minimization famously relies on this
\cite{brzozowski1962}.)

The result in either case is that the associated regular language is the reverse
of the original (in the sense that each string in the set is reversed).  In the
same fashion, we show that corresponding operations exist for \NetKAT Expressions
and \NetKAT Automata. Later, we will show these operation can be used to tie
together the mechanisms of two bisimulation algorithms.

\subsection{Reversing \NetKAT Expressions}\label{sec:rev-exp}

Given a guarded string $w \in \GS$, where
\[w = \pk\cdot \pk_1 \cdot \dup \cdot \ldots \cdot \pk_n \ddd \pkp,\] we define its \emph{reverse} to be
\[w^R \triangleq \pkp \cdot \pk_{n} \ddd \ldots \cdot \pk_1 \ddd \pk.\]

We extend this to sets of guarded strings in the natural way, $L^R \triangleq \{w^R \mid w \in L\}$.
Given a \NetKAT expression, we can recursively compute its reverse in a
similar manner to reversing a regular expression. The construction is given
in \Cref{fig:reverse-exp}.

\begin{figure}
\begin{tabular}{|c|c|}
    \hline
    $p\in\Exp$ & $p^R\colon \Exp$ \\
    \hline
    $\zero$ & $\zero$ \\
    $\one$ & $\one$ \\
    $\dup$ & $\dup$ \\
    $f = v$ & $f = v$ \\
    $f \gets v$ & $f = v \cdot \sum_{v'} f \gets v'$ \\
    $q + r$ & $q^R + r^R$ \\
    $q \cdot r$ & $r^R \cdot q^R$ \\
    $q^\star$ & $(q^R)^\star$ \\
    \hline
\end{tabular}
    \caption{Inductive definition of the reverse operation for \NetKAT
    expressions. See \Cref{sec:set-all} for more on the reverse of assignment.}
\label{fig:reverse-exp}
\end{figure}

The interesting cases in \Cref{fig:reverse-exp} are tests, i.e., $f = v$, and
sets, i.e., $f \gets v$. The subtlety of the fact that tests are their own
reverse is that the same expression acting as a filter in each
direction is actually symmetric. In a similar vein, the reverse of an assignment
follows the intuition that an assignment transforms a packet with any value for
$f$ into one with the specified value---so the reverse must check the specified
value and emit packets with all possible values. 

First we show the correctness of reversal. That is, the guarded strings set of
reverse of a \NetKAT expression is the set of guarded strings of the original
expression, reversed:

\begin{theorem}
    For any \NetKAT expression $p$, we have $\Sem{p^R} = \Sem{p}^R$
\end{theorem}
\begin{proof}
    By induction on the structure of $p$. If $p$ is $\zero$, $\one$, $f=v$, or
  $\dup$, the statement holds immediately, since in each case, $p = p^R$, and
  the associated packet histories are palindromes.

    Suppose $p = (f \gets v)$.

    For the inductive cases, assume the theorem statement for subexpressions $q$, $r$:
\begin{enumerate}[(a)]
    \item $p = q + r$. Then we have:
    \begin{align*}
        \Sem{p^R} &= \Sem{q^R + r^R} \\
               &= \Sem{q^R} \cup \Sem{r^R} \\
               &= \Sem{q+r}^R \\
               &= \Sem{p}^R
    \end{align*}
    \item $p = q \cdot r$. Then $\Sem{p^R} = \Sem{r^R\cdot q^R}$. Let 
    $w \in \Sem{r^R\cdot q^R}$. Then 
    \[ w = \pk \cdot \pk_1 \cdot \dup \cdot \ldots \cdot \dup \cdot \pk_i \diamond \pk_i
    \cdot \dup \cdot \ldots \cdot \pk_n \cdot \dup \cdot \pkp.\]
    But this means
    \[ w^R = \pkp \cdot \pk_n \ddd\ldots\ddd \pk_i \diamond \pk_i \cdot\dup\cdot\ldots\cdot\dup\cdot\pk_1\cdot\pk. \]
    and so $w^R \in \Sem{q\cdot r}$, implying $w \in \Sem{p}^R$.
    \item $p = q^\star$. Follows by induction on the length of the guarded string,
    applying the previous two cases.
\end{enumerate}
\end{proof}

\subsubsection{Difficulty with Set-All}\label{sec:set-all}
The reader may have noticed that the reverse of an assignment
presents a few challenges.
First, to even write down the expression, we are forced to specify explicitly the
set of values the field ranges over. This is particularly inconvenient, because
to leave these sets unspecified allows gives the expressions better
composability.
Second, the size of the sum is exponential in the bit width of the field and
would be unwieldy if treated naively.

Both of these issues can be neatly addressed by dealing with the ``set all''
assignment by choosing a \emph{symbolic} representation, in a
manner following existing work on symbolic analysis of automata
\cite{pous2015,dantoni-veanes2014,dantoni-veanes2017}. The details are given in the description of our
implementation in \Cref{sec:grabbag}.

Even without the trick of using symbolic methods, it often turns out that the
set-all can often be canceled in practice by reasoning using KAT axioms.
An example presents itself as we write out the check that $f \gets v \equiv {(f \gets v)^R}^R$ (as
we would hope!). We have:
\begin{align*}
    {(f \gets v)^R}^R &\equiv (f=v \cdot f=v \cdot \sum_{v'} f \gets v')^R & \text{Definition of reverse} \\
                      &\equiv \sum_{v'} f=v' \cdot (\sum{v''} f\gets v'')  \cdot
                      f=v & \text{Definition of reverse} \\
                      &\equiv  f\gets v & \textsc{PA-Mod-Filter}\text{ and }\textsc{PA-Contra}
\end{align*}

In a frequently occuring case, a test followed by assignment to the same field,
such as
$sw=1 \cdot sw \gets 2$
naturally has the reverse (after cancelling similarly):
$sw = 2 \cdot sw \gets 1$
and the set-all does not appear.

In general, given an expression $e$, as long as every packet field is tested
before it is assigned to (respecting the paths through the expression), then
there is an expression without set-all equivalent to $e^R$. \mm{MM formalize a
bit more?} Note that this set-free expression may
require more extensive ``sliding'' of tests. For example, given:
\[sw=1 \cdot \dup \cdot \dup \cdot sw \gets 2\]
naively applying reverse becomes:
\[sw=2\cdot \sum_{v'} sw=v'\cdot sw\gets v' \cdot \dup \cdot \dup \cdot sw=2\]
However after two applications of \textsc{PA-Dup-Filter-Comm}, we get that:
\[sw=2 \cdot sw\gets 1 \cdot \dup \cdot \dup \]
is equivalent.

\subsubsection{Example}

The reverse expression of $e_1$ is:

\[ e_1^R \triangleq x=1 + (y=1\cdot \sum_v y\gets v\ddd z=4 \cdot z \gets 2 +
        y=3\cdot \sum_v y\gets v \cdot z=1) \ddd y=7 \cdot y\gets 2
\]

Applying \textsc{PA-Dup-Filter-Comm},
         \textsc{PA-Mod-Filter-Comm}, and
         \textsc{BA-Seq-Comm}, we get a set-all-free expression:

\[ e_1^R\ \equiv x=1 + (y=1\cdot y\gets 7 \ddd z=4 \cdot z \gets 2 +
        y=3\cdot y\gets 7 \cdot z=1) \ddd y\gets 2
\]

\subsubsection{Network Policy Example}\label{sec:rev-fast-example}

We now construct an example for which having a network policy reverse operation
would improve analysis time. We draw inspiration from the classic example of
a family of regular expressions\footnote{Here exponentiation by $n$ is just syntactic sugar for
$n$-fold concatenation} whose conversion to deterministic automaton
results in exponential space in $n$, $(a+b)^\star a(a+b)^n$
\cite{hopcroft-automata}.

Suppose we have a \NetKAT policy $\mathsf{p}$ written for a topology
$\mathsf{t}$ (also expressed in \NetKAT), and input and output predicates of
interest $\mathsf{in}$ and $\mathsf{out}$.

The usual way to combine these into a single expression, for example to check that $\mathsf{out}$ is
reachable from $\mathsf{in}$, is as follows \cite{netkat}:

\[
\mathsf{in}\cdot(\mathsf{p}\cdot\mathsf{t}\cdot\dup)^\star\cdot\mathsf{out} \nequiv \zero
\]

Suppose instead that we want to check that the $n^{th}$-to-last switch traversed is always
a particular one, say $\mathsf{sw=3}$. We can express this property as a \NetKAT equivalence
query by writing:

\[
\mathsf{in}\cdot(\mathsf{p}\cdot\mathsf{t}\cdot\dup)^\star\cdot\mathsf{out} \nequiv
\mathsf{in}\cdot(\mathsf{p}\cdot\mathsf{t}\cdot\dup)^\star\cdot\mathsf{sw}=3\cdot
(\mathsf{p}\cdot\mathsf{t}\cdot\dup)^n\cdot\mathsf{out}
\]

While the \NetKAT automaton on the righthand only has $n$ additional states
using the construction from \cite{coalgebraic}, when determinizing to check
equivalence, the state-packet-space blowup is exponential just as the example
is as a deterministic finite automaton.

For this instance, we can check this property much faster by reversing both expressions first.
The result is that the resulting deterministic automata have size linear in the size of the original
expression and are analyzed faster accordingly.

\mm{it would be best to have a concrete instance with empirical data of this.}

\subsection{Reversing \NetKAT Automata}

We now give the construction for reversing \NetKAT automata. As with \NetKAT
expression reversal, the key idea comes from the regular language analogue,
however here the care must be taken to ensure that the packet updates are
appropriately handled.

Given a \NetKAT automaton $\mathcal{A} = (S, s_0, \epsilon, \delta)$, we define its
reverse $\mathcal{A}^R = (S, s_0, \epsilon^R, \delta^R)$, where
\footnote{Parameter names $\pkp$ and $s'$ are chosen to preserve the intuition that
transitions \emph{in $\mathcal{A}$} are from $s$ to $s'$ and $\pk$ to $\pkp$.}:
\[ \epsilon^R_{\pkp}(s') \triangleq \begin{cases}
    \{\pk\in\Pk \mid \pkp \in \epsilon_\pk(s_0)\}  & s' = s_0  \\
    \{\pk\in\Pk \mid s' \in \delta_{\pk\pkp}(s_0)\} & \text{otherwise}
\end{cases}\]
and
\[ \delta^R_{\pkp}(s') \triangleq \begin{cases}
  \{(\pk,s) \in \Pk\times S \mid \pkp \in \epsilon_\pk(s)\} & s' = s_0 \\
    \{(\pk,s) \in \Pk\times S \mid s' \in \delta_{\pk\pkp}(s)\} & \text {otherwise}
                    \end{cases} \]

For $\varepsilon^R$, the two cases are that (a) the observation function
\emph{for the start state} should be inverted directly, and (b) for other
states, the observation function should invert transitions out of the original start state.

For $\delta^R$, the two cases are that (a) transitions out of the new start state
come from inverting the observation function for each original (non-start) state, and (b) the
other transitions (i.e., not out of the start state) are just inverted.

\begin{lemma}\label{lem:rev-rev-id}
Let $\mathcal{A}$ be a \NetKAT automaton. Then $\mathcal{A}$ is precisely the
same automaton as ${\mathcal{A}^R}^R$.
\end{lemma}
\begin{proof}
Immediate, by inspection.
\end{proof}


As we would expect, reversing a \NetKAT automaton reverses all of the guarded
strings it accepts. To prove this, first we need a lemma that reversing an
automaton reverses the paths through it\footnote{Note that for $w\in(\Pk\cdot\dup)^\star$, we have
$w^R\in(\Pk\cdot\dup)^\star$, \emph{not} $w^R\in(\dup\cdot\Pk)^\star$. The ``reverse and
slide dups'' here is consistent with the convention for reversing guarded
strings given in \Cref{sec:rev-exp}.}.
\begin{lemma}\label{lem:paths-reverse}
Let $\mathcal{A} = (S, s_0, \epsilon, \delta)$ be a \NetKAT automaton, $s\in S$ a
state. Then:
\begin{enumerate}[(a)]
\item If $s \neq s_0$ and $w = \pk_0\cdot w' \in \Pk\cdot(\Pk\cdot\dup)^\star$,
                then for each $(\pkp, s') \in \delta^\star_w(s)$, we have
                $(\pk_0, s)\in {\delta^R}^\star (s') (\pkp\cdot w^R$).

\item If $w = \pk_0\cdot \pk_1\ddd w'\in\Pk\cdot\Pk\ddd(\Pk\cdot\dup)^\star$,
                then for each $(\pk, s) \in \delta^\star_w (s_0)$,
                we have $\pk\cdot w^R \in \accept_{\mathcal{A}^R}(s)$.
\end{enumerate}
\end{lemma}
\begin{proof}\ \\
\begin{enumerate}[(a)]
\item Let $s \neq s_0$ and $w = \pk_0\cdot w' \in \Pk\cdot(\Pk\cdot\dup)^\star$.
We proceed by induction on the length of $w$.
In the base case, we have $w = \pk_0$ and $w' = \varepsilon$. But that means that
    $\delta^\star_w (s) = \{(\pk_0, s)\}$, and so $(\pk_0, s) \in {\delta^R}^\star(s)(\pk_0)
= \{(\pk_0,
s)\}$ as needed. In the induction case, let $w = \pk_0\cdot \pk_1 \ddd w'$, and let
$(\pkp, s') \in \delta^\star_w (s)$. This means there is a $s''\in S$
    such that $s''\in\delta_{pk_0,pk_1} (s)$ and $(\pkp,s')\in\delta^\star (s'') (\pk_1\cdot w'$).
    Using the induction hypothesis, we get that $(\pk_1, s'')\in{\delta^R}^\star(s') (\pkp\cdot w'^R\cdot \pk_1)$.
    By the definition of $\delta^R$ and that $s''\in\delta_{\pk_0\pk_1} (s)$, we know
    $s\in\delta^R_{\pk_1\pk_0} (s'')$. Putting these together we have shown that
$(\pk_0, s)\in{\delta^R}^\star\ s'\ (\pkp\cdot w'^R \cdot \pk_1 \cdot\dup)$ as required.

\item Let $w = \pk_0\cdot \pk_1\ddd w'\in\Pk\cdot\Pk\ddd(\Pk\cdot\dup)^\star$ and
  let $(\pk, s)\in \delta^\star_w (s_0)$.
Taking one step forward from $s_0$, we find a state $s_1\in S$ such that $(\pk_1,
s_1) \in \delta\ s_0\ \pk_0$
    and for which $(\pk, s)\in \delta^\star (s_1) (\pk_1\cdot w')$. Using (a), this
means $(\pk_1,
    s_1) \in {\delta^R}^\star (s) (\pk\cdot w'^R)$. By the definition of $\epsilon^R$, we
    know $\pk_0 \in \epsilon^R_{\pk_1} (s_1)$. Putting these together
    (and since $w'^R \cdot \pk_1\ddd \pk_0 = w^R$), we have shown as
    needed that $\pk\cdot w'^R\cdot \pk_1\ddd \pk_0 \in \accept_{\mathcal{A}^R}(s)$.

\end{enumerate}
\end{proof}

Now we have the desired result:
\begin{theorem}
    For a \NetKAT Automaton $\mathcal{A} = (S, s_0, \epsilon, \delta)$, we have
    $L(\mathcal{A})^R = L(\mathcal{A}^R)$.
\end{theorem}
\begin{proof}
    ($\subseteq$). First we show that $L(\mathcal{A}) \subseteq L(\mathcal{A}^R)^R$. Since it is
    the case that both (a) $L \subseteq L'$ iff $L^R \subseteq L'^R$ and (b)
    ${L^R}^R = L$, this will imply $L(\mathcal{A})^R \subseteq L(\mathcal{A}^R)$.
    Let $w \in L(\mathcal{A})$. We consider two cases.
    \begin{enumerate}[(a)]
        \item $w = \pk_{in}\cdot \pk_{out}$. Then it must be that $\pk_{out}\in\epsilon\ s_0\
        \pk_{in}$. By construction, $\pk_{in}\in\epsilon^R\ s_0\ \pk_{out}$, implying
        $w^R \in L(A^R)$.
        \item $w = \pk_{in}\cdot \pk_1\ddd w'\cdot \pk_{out}$ for some $w'\in(\Pk\cdot\dup)^\star$.

           From $w\in L(\mathcal{A})$, we know there is a $(\pk,s)\in \Pk\times S$ such
           that $(\pk, s)\in \delta^\star\ s_0\ \pk_{in}\cdot \pk_1\ddd w'$ and
           $\pk_{out}\in \epsilon\ s\ \pk$. Applying \Cref{lem:paths-reverse}(b), we
           know $\accept_{\mathcal{A^R}}\ s\ (\pk\cdot w^R)$. By
           construction of $\delta^R$, we see that $\delta^R\ s_0\ \pk_{out} =
           (\pk, s)$.
           Combining, we get $\accept_{\mathcal{A^R}}\ s_0\ w^R$, and
           thus that $w^R \in L(\mathcal{A}^R$).

    \end{enumerate}
    ($\supseteq$). We get the other direction for free, since we can make the same
    argument starting with $\mathcal{A}^R$ (and applying \Cref{lem:rev-rev-id}).
\end{proof}

\subsection{Example}

Consider the automaton we built above for $e_1$. Here is its reverse (after
sliding tests to eliminate set-all):
\begin{center}
\begin{tikzpicture}
\node[state, initial] (s0) {s0};
\node[state, right of=s0] (s1) {s1};
\node[state, right of=s1] (s2) {s2};
\node[draw=none, below of=s0, node distance=27pt] {$x=1$};
\node[draw=none, below of=s1, node distance=27pt] {$0$};
\node[draw=none, below of=s2, node distance=27pt] {$y\gets 2$};
\draw (s0) edge node{$y=1\cdot y\gets 7$} (s1);
\draw (s0) edge[ematrix] ++ (0,-0.75);
\draw (s1) edge[ematrix] ++ (0,-0.75);
\draw (s1) edge node{$z=4 \cdot z\gets 2$}  (s2);
\draw (s2) edge[ematrix] ++ (0,-0.75);
\draw (s0) edge[bend left=50] node{$y=3\cdot z=1\cdot y\gets 7$} (s2);
\end{tikzpicture}
\end{center}

One can check that the same automaton is obtained either by (1) constructing the
automaton above and (2) performing automaton reversal, or by (1) reversing the
expression to get $e_1^R$, and (2) constructing the automaton for $e_1^R$.




\section{text deleted}

\as{move this remark elsewhere -- also all text below remark, unclear it belongs here}
\begin{remark}
For this paper, we will disallow transitions from returning to the start state
as a convenience for several of the constructions.
That is, the continuation function actually has type $\delta\colon S \to \Pk \to
\mathcal{P}(\Pk \to S - \{s_0\})$. This is no real limitation, of course,
because for any automaton with a transition to the start state, we can simply
copy the start state and transition to the copy instead.
\end{remark}
As a convenient shorthand, whenever we have a set $X
\subseteq \Pk \times S$ of packet-state pairs, it will be convenient to ``select'' the set of states associated with a
particular packet. To denote this for such an $X$ and packet $\pk$, we use a restriction bar:

\[ X \vert_\pk \triangleq \{ s \in S \mid (\pk, s) \in X \} \]


Note that the sequences processed by $\delta^\star$ are not guarded strings,
because they omit the final transformation that would be performed by the
observation function. Thus, an equivalent definition for $\accept$ would
be, for a $w \in \Pk\cdot(\Pk\cdot\dup)^\star$:
\[\accept_\mathcal{A}\ s\ w\cdot \pk_{out} \triangleq \pk_{out} \in 
    \bigcup_{(\pk, s) \in \delta^\star\ s_0\ w} \epsilon\ s\ \pk \]







\subsection{\NetKAT Automata (non-det)}
The (denotational) semantics of \NetKAT expressions is given in terms of regular languages of 
guarded strings. These strings are inputs to an operational model introduced in~\cite{fastcompiler}, akin to a non-deterministic automaton. 
\begin{definition}A {\em \NetKAT automaton} is a tuple $(S, s_0, \epsilon, \delta)$, where:
\begin{itemize}
    \item $S$ is a finite set of states.
    \item $s_0 \in S$ is the start state.
    \item $\epsilon\colon S \to \Pk \to \pow\Pk$ is the observation
    function.
    \item $\delta\colon S \to \Pk \to\pow{\Pk \times S}$ is the
    continuation function.
\end{itemize}
\end{definition}

We will sometimes use equivalent presentations of $\epsilon$ and $\delta$ as families of functions over $\Pk$:
\[
\begin{array}{c}
\epsilon\colon S \to \Pk \to \mathcal{P}(\Pk) \\
\hline\hline
 \epsilon_\pk \colon S \to \mathcal{P}(\Pk)\\\\
\end{array}\qquad
\begin{array}{c}
\delta\colon S \to \Pk \to \mathcal{P}(\Pk \times S) \\
\hline\hline
\delta_\pk \colon S \to \mathcal{P}(\Pk \times S)\\
\hline\hline
\delta_{\pk\pkp} \colon S \to \mathcal{P}(S)\\
\end{array}
\qquad
\begin{array}{l@{\,}c@{\,}l}
  \epsilon_\pk (s) &=&   \epsilon(s)(\pk)\\
\delta_\pk (s) &=& \delta(s)(\pk)\\
\delta_{\pk\pkp}(s) &=& \{s' \mid (\pkp,s')\in \delta (s)(\pk) \}
\end{array}
 \]
To process a string $\pk \cdot \pk_1 \cdot \dup \cdot \ldots \cdot \pk_n \cdot \dup \cdot \pkp$, for $n\geq 0$, an automaton in a state $s\in S$ can either accept the string if $n=0$ and $\pkp \in \epsilon(s)(\pk)$ or, if $n>0$,  it can consume one packet and $\dup$ from the start of the string if $(\pk_1,s')\in\delta(s)(\pk)$ and transition to state $s'$ in which the rest of the string needs to be processed. More formally, the set of strings accepted by a state $s\in S$ is formalized in the function 
    $\accept\colon S \to \pow\GS $ given by:
    \begin{align*}
     \pk \cdot \pkp \in   \accept(s) &\iff \pkp \in \epsilon_\pk(s) \\
     \pk \cdot \pkp \cdot \dup \cdot w \in   \accept(s) &\iff \exists s' \in \delta_{\pk\pkp}(s) \text{ such that } 
            \pkp \cdot w \in \accept(s') \\
    \end{align*}
   We can then define the set of strings accepted by an automaton $\mathcal A= (S, s_0, \epsilon, \delta)$ as:
\[L(\mathcal{A}) \triangleq \accept(s_0) \]

%%%%%%%%%%%%%%%%%%%%%%%%%%%%%%%%%%%%%%%%%%%%%%%%%%%%%%%%%%%%%%%%%%%%%%%%%%%%%%%%%%%%%%%%%%%%%
% MM: I added back the definition for δ* because I need it for proof of
% \cref{lem:paths-reverse} (and follow-on theorems)
The transition function can be extended to sequences of packets having type
$\delta^\star\colon S \to \Pk\cdot(\Pk\cdot\dup)^\star\to\mathcal{P}(\Pk\times S)$. This usage
is analogous to the standard extension of Brzozowski derivative to strings. We use it here to describe
paths through automata. We define this extension as:
\begin{equation*}
  \begin{aligned}
    \delta^\star(s)(\pk) &\triangleq \{(\pk, s)\}\\
    \delta^\star (s) (\pk\cdot\pkp\ddd w) &\triangleq \bigcup_{s' \in \delta_{\pk\pkp}(s)} \delta^\star(s') (\pkp\cdot w)
\end{aligned}\qquad
\begin{array}{c}
\delta^\star_w (s) = \delta^\star(s)(w)
\end{array}
\end{equation*}
%%%%%%%%%%%%%%%%%%%%%%%%%%%%%%%%%%%%%%%%%%%%%%%%%%%%%%%%%%%%%%%%%%%%%%%%%%%%%%%%%%%%%%%%%%%%

\begin{remark}[Deterministic vs non-deterministic \NetKAT automata]
The above model of \NetKAT automata is non-deterministic as each transition in $\delta$ can lead to multiple next states for the same input packet. When we define the $\accept$ function we implicitly construct (on-the-fly) a determinization of the automaton as we only look for one accepting path labelled by a string. In the coalgebraic decision procedure presented in~\cite{coalgebraic} the authors considered a deterministic model, as this gives a canonical correspondence with the denotational semantics. The main advantage of taking a  non-deterministic model as starting point is that one can build more compact automata corresponding to a NetKAT program and, as we will see later, we can still design the decision procedures in an efficient way. 

From any \NetKAT automaton  $(S, s_0, \epsilon, \delta)$ we can build a deterministic \NetKAT automaton  $(\pow S, \{s_0\}, \widehat\epsilon, \widehat\delta)$ where:
    \begin{align*}
&\widehat\epsilon \colon \pow S \to \Pk \to \pow\Pk &&\widehat \delta\colon \pow S \to \Pk \to\pow{\Pk \times S}\\
&\widehat\epsilon (U) = \bigcup\limits_{s\in U} \epsilon_\pk(s)&&\widehat\delta (U)(\pk) = \bigcup\limits_{s\in U} \delta_\pk(s)
\end{align*}
\end{remark}

\subsection{Bisimulations of \NetKAT automata (non-det)} 

In this section we present a first naive algorithm to check equivalence of two \NetKAT automata. 


\begin{definition}[Bisimulation]
    A \emph{bisimulation} between \NetKAT automata $\mathcal{A} = (S, s_0, \epsilon, \delta)$
    and $\mathcal{B} = (T, t_0, \epsB, \delB)$ is a relation $ R
    \subseteq \mathcal{P}(S) \times \mathcal{P}(T)$ such that $\{s_0\} R \{t_0\}$ and for all $U\subseteq S$ and $V\subseteq T$, if $(U,V)\in R$ then:
\begin{enumerate}[(a)]
    \item For all $\pk \in\Pk$, $\widehat\epsilon_\pk(U) = \widehat{\epsB}_\pk(V)$ \text{ and }
    \item For all $\pk,\pkp \in\Pk$, ($\widehat\delta_{\pk\pkp}(U), \widehat{\delB}_{\pk\pkp}(V))\in R$.
\end{enumerate}
\end{definition}
In \Cref{fig:naivend} we present a first algorithm to compute a bisimulation between two \NetKAT automata. Bisimulations are sound proof principles for language equivalence of \NetKAT automata. 




\begin{figure}
\begin{algorithmic}
\State $R\gets \emptyset; \todoR \gets \{(\{s_0\},\{t_0\})\}$, 
\While{$\todoR$ is not empty} 
    \State extract $(X, Y)$ from $\todoR$;
    \If{$(X, Y) \not\in R$} 
     \If{$\widehat\epsilon(X) \neq \widehat\epsilon(Y)$} 
      \Return \false;
	\EndIf 
      \For {$\pk,\pkp \in \Pk$}
     	insert ($\widehat\delta_{\pk\pkp}(X)$,$\widehat\delB_{\pk\pkp}(Y)$) in \todoR;
\EndFor
    \EndIf
     insert $(X,Y)$ in $R$;
\EndWhile
\State 
\Return \true;
\end{algorithmic}

%\begin{lstlisting}[language=Python,mathescape]
%$R$ := { }; todo := {({$s_0$}, {$t_0$})};
%while todo is not empty, do {
%     extract $(X, Y)$ from todo;
%     if $(X, Y)\in R$ then skip;
%     if $\widehat\epsilon(X) \neq \widehat\epsB(Y)$ then return false;
%     for all $\pk,\pkp \in \Pk$ 
%     	insert ($\widehat\delta_{\pk\pkp}(X)$,$\widehat\delB_{\pk\pkp}(Y)$) in todo
%     insert $(X,Y)$ in $R$
%return true;
%\end{lstlisting}

\caption{Naive bisimulation}\label{fig:naivend}
\end{figure}


\begin{theorem}[Soundness]
If there exists a bisimulation $R$ between \NetKAT automata $\mathcal{A} = (S, s_0, \epsilon, \delta)$
    and $\mathcal{B} = (T, t_0, \epsB, \delB)$ then $L(\mathcal{A}) = L(\mathcal{B})$.
\end{theorem}


\begin{example}
Let us consider the following two \NetKAT automata, which for simplicity operate over packets with one field $x$. 
\begin{center}
\begin{tikzpicture}[node distance =2.5cm] \tikzstyle{every node}=[font=\footnotesize]
\node[state, initial] (s0) {$p$};
\node[state, right of=s0] (s1) {$q$};
\node[state, right of=s1] (s2) {$r$};
\node[draw=none, below of=s0, node distance=27pt] {$\zero$};
\node[draw=none, below of=s1, node distance=27pt] {$\zero$};
\node[draw=none, below of=s2, node distance=27pt] {$x=1$};
\draw (s0) edge[bend left] node{$x=2, x= 1$} (s2);
\draw (s0) edge[ematrix] ++ (0,-0.75);
\draw (s1) edge[ematrix] ++ (0,-0.75);
\draw (s1) edge node{$x=2, x= 1$}  (s2);
\draw (s2) edge[bend left] node{$x=2, x= 1$}  (s1);
\draw (s1) edge node{$x=2, x= 1$}  (s0);
\draw (s2) edge[ematrix] ++ (0,-0.75);
\end{tikzpicture}\qquad 
\begin{tikzpicture}[node distance =2.5cm] 	 \tikzstyle{every node}=[font=\footnotesize]
\node[state, initial] (s0) {$s$};
\node[state, right of=s0] (s1) {$t$};
\node[state, right of=s1] (s2) {$u$};
\node[draw=none, below of=s0, node distance=27pt] {$\zero$};
\node[draw=none, below of=s1, node distance=27pt] {$\zero$};
\node[draw=none, below of=s2, node distance=27pt] {$x=1$};
\draw (s0) edge[bend left] node{$x=2, x= 1$} (s2);
\draw (s0) edge[ematrix] ++ (0,-0.75);
\draw (s1) edge[ematrix] ++ (0,-0.75);
\draw (s2) edge node{$x=2, x= 1$}  (s1);
\draw (s0) edge[bend right] node[below]{$x=2, x= 1$}  (s1);
\draw (s1) edge node[above]{$x=2, x= 1$}  (s0);
\draw (s2) edge[ematrix] ++ (0,-0.75);
\end{tikzpicture}
\end{center}
We depict initial states with an incoming arrow, observation functions by double arrows on each state to a predicate $a$ which is denoting the set of packets $\pSem a$. The continuation functions are depicted as follows: $s \xrightarrow{a,b} s'$  iff for all $\pk\in \pSem a$ and $\pkp\in \pSem b$ we have $s' \in \delta_{\pk\pkp}(s)$. 

The naive bisimulation algorithm in ~\Cref{fig:naive} will build a relation $R$ between states of the determinized automata, containing the pairs connected below through dashed lines: 

\begin{tikzpicture}[node distance =1.5cm] \tikzstyle{every node}=[font=\footnotesize]
\node (s0) {$\{p\}$};
\node[right of=s0] (s1) {$\overline{\{r\}}$};
\node[right of=s1] (s2) {$\{q\}$};
\node[right of=s2] (s3) {$\overline{\{p,r\}}$};
\node[right of=s3] (s4) {$\overline{\{q,r\}}$};
\node[right of=s4] (s5) {$\overline{\{p,q,r\}}$};
\node[below of=s0, node distance=1cm] (t0) {$\{s\}$};
\node[right of=t0] (t1) {$\overline{\{t,u\}}$};
\node[right of=t1] (t2) {$\{s,t\}$};
\node[right of=t2] (t3) {$\overline{\{s,t,u\}}$};
\draw (s0) edge  node{$\ell$} (s1);
\draw (s1) edge  node{$\ell$} (s2);
\draw (s2) edge  node{$\ell$} (s3);
\draw (s3) edge  node{$\ell$} (s4);
\draw (s4) edge  node{$\ell$} (s5);
\draw (s5) edge[out=25, in=365,loop]  node{$\ell$} (s5);
\draw (t0) edge  node{$\ell$} (t1);
\draw (t1) edge  node{$\ell$} (t2);
\draw (t2) edge  node{$\ell$} (t3);
\draw (t3) edge[out=365, in=345,loop]  node{$\ell$} (t3);
\draw (s0) edge [-,dashed]  (t0);
\draw (s1) edge [-,dashed]  (t1);
\draw (s2) edge [-,dashed]  (t2);
\draw (s3) edge [-,dashed]  (t3);
\draw (s4) edge [-,dashed]  (t3);
\draw (s5) edge [-,dashed]  (t3);
\end{tikzpicture} 

We use $\ell$ as a shorthand to denote $x=2,x=1$ and the overline as a shorthand
to indicate observation $x=1$. It is worth noting that because we are
determinizing on-the-fly and building a relation over an automaton with a
structured state space (sets) we can short-circuit the construction of the
bisimulation in the third step as we can infer the last three pairs from the
first three (this is Bonchi-Pous bisimulation up-to congruence
technique~\cite{bonchi-pous2013}).
\end{example}
The bisimulation construction of \Cref{fig:naive} can be optimized in different ways and this will result in the different algorithms we present later in the paper. Note how the for-loop guard ${\pk,\pkp \in \Pk}$ results in the insertion of new pairs in the worklist $\todoR$ that grow exponentially with the size of basic tests (or number of fields). Efficient encoding of the transitions $\delta_{\pk,\pkp}$ and the bisimulation relations on these will result in smaller relations and more efficient checks of equivalence. We will use a data structure introduced in \cite{fastcompiler}---forward-decision diagrams---which are a mild generalization of binary-decision diagrams. 


\subsection{Conversion of \NetKAT Expressions to Automata (non-det)} 
In order to apply the bisimulation algorithms to \NetKAT programs we need to convert expressions to automata. We present a direct construction of a non-deterministic automaton using an analogue of Antimirov derivatives~\cite{antimirov}. We first show that the set $\Exp$ has the structure of a $\NetKAT$ coalgebra, that is a \NetKAT automaton without an initial state and with an infinite state space. We then prove that for a fixed program $p$ the subcoalgebra starting at $p$ has a finite state space. 

%This is what the compiler paper does -- if the implementation does this I need to backtrack this explanation (Alexandra)
% For an expression with $n$ occurrences of $\dup$, we will build an automaton that has $n+1$ states---this is similar to the construction due to Berry-Sethi and Glushkov~\cite{berri-sethi,glushkov} that builds a position automaton corresponding to a regular expression. 

A \NetKAT coalgebra is a pair $(S,\left<\epsilon,\delta\right>)$, where $S$ is a (potentially infinite) set of states and $\epsilon,\delta$ have the same types as in a \NetKAT automaton. Given $T\subseteq S$ we say that $(T,\left<\epsilon\!\!\mid_T,\delta\!\!\mid_T\right>)$ is a subcoalgebra of $(S,\left<\epsilon,\delta\right>)$ if the restricted transition function is well-defined (in other words, $T$-transitions stay in $T$). We can define the state space of the subcoalgebra generated by a state $s\in S$, denoted $\left<s\right>$ as the intersection of all subcoalgebras containing $s$:
\[
\bigcap \{ T \mid (T,\left<\epsilon\vert_T,\delta\vert_T\right>) \text{ is a subcoalgebra of } (S,\left<\epsilon,\delta\right>) \text{ and } s\in T \}
\]
Intuitively, this definition is capturing the construction of a coalgebra starting from $\{s\}$ and closing it under repeatedly taking transitions through $\delta$. 

\begin{definition} The \NetKAT automaton $\mathcal A_p$ for a program expression $p\in\Exp$ is the automaton $(S, s_0, \epsilon, \delta)$, where:
\begin{itemize}
    \item $S = \left<p\right> \subseteq \Exp$ (by taking the subcoalgebra of $(\Exp, \left<E,D\right>)$).
    \item $s_0 = p$ is the start state.
    \item $\epsilon\colon S \to \Pk \to \mathcal{P}(\Pk)$ is given by $\epsilon(s)(\pk) = E_\pk(p)$.
        \item $\delta\colon S \to \Pk \to \mathcal{P}(\Pk \times S)$ is given by $\delta(s)(\pk) = D_\pk(p)$.
\end{itemize}
\end{definition}
To ensure the soundness of the above definition we need to show that $S$ is a finite set; that is, the set of expressions obtained by repeatedly applying $D$ to $p$ is finite.
\begin{theorem}
The set of $\Exp$ has the structure of a \NetKAT coalgebra, given by the functions $E\colon \Exp \to \Pk \to \mathcal{P}(\Pk)$ and $D \colon\Exp \to \Pk \to \mathcal{P}(\Pk \times \Exp)$ in \Cref{fig:derivatives}. Moreover, for every $p\in\Exp$ the subcoalgebra generated by $p$ is finite. 
\end{theorem}
\begin{proof}
This result follows from \Cref{lemma:derivatives}: if $(\pkp, q) \in D_\pk(p)$, then $D_{\pkp}(q) \subseteq D_\pk(p)$. The number of derivatives is bounded by $|\Pk| \times \ell$, where $\ell$ is the number of occurrences of $\dup$ and $+$ in an expression. 
\end{proof}
\begin{figure}



\begin{tabular}{|c|c|c|}
    \hline
    $p \in \Exp$        & $E_\pk \colon\pow\Pk$            & $D_\pk \colon \Exp\to\pow{\Pk \times \Exp}$ \\
    \hline
    $\zero$                   & $ \emptyset$                                          & $\emptyset$ \\
    $\one$                  &  $\{\pk\}$                                            & $\emptyset$ \\
    $f = n$             & $\{\pk\mid \pk.f = n\}$                         & $\emptyset$  \\
    $f \gets n$         & $\{\pk[n/f]\}$                     & $\emptyset$ \\
     $\dup$ &  $\emptyset$                                            & $\{(\pk, \one)\}$\\
    $q + r$             & $E_{\pk}(q) \cup E_{\pk}(r)$     & $D_{\pk}(q) \cup D_{\pk}(r)$\\
    $q \cdot r$         & $\bigcup\limits_{\pkp\in E_\pk(q)}  E_{\pkp}(r)$ & $D_{\pk}(q)\fatsemi r \cup \bigcup\limits_{\pkp\in E_{\pk}(p)} D_{\pkp}(r)$\\
    $q^\star$           & $\{\pk\} \cup \bigcup\limits_{\pkp\in E_\pk(q)}  E_{\pkp}(q^\star)$                    &  $D_{\pk}(q)\fatsemi q^\star \cup \bigcup\limits_{\pkp\in E_{\pk}(q)} D_{\pkp}(q^\star)$ \\
    \hline
\end{tabular}
    \caption{\NetKAT coalgebra via Antimirov derivatives. Note that the definitions for iteration are circular, but both are well-defined if we take the least fixpoint of the system of equations. The operation $\fatsemi \colon \pow{\Pk \times \Exp} \times \Exp \to \pow{\Pk \times \Exp}$ is defined as $U \fatsemi q = \{(\pk, p;q) \mid (\pk,p)\in U\}$. }
    \label{fig:derivatives}
\end{figure}


To illustrate the construction, consider the following \NetKAT program expression $p$:

\[
p \triangleq
    x=1 +  y=2\cdot y\gets7\cdot\dup\cdot(z=1\cdot y\gets3 +  z=2\cdot z\gets4\cdot\dup\cdot y\gets1)
\]

The automaton for $p$ obtained will have $3$ states (we omit states with an empty output):
\begin{center}
\begin{tikzpicture}
\node[state, initial] (s0) {$p$};
\node[state, right of=s0] (s1) {$q$};
\node[state, right of=s1] (s2) {$r$};
\node[draw=none, below of=s0, node distance=27pt] {$x=1$};
\node[draw=none, below of=s1, node distance=27pt] {$z=1\cdot y= 3$};
\node[draw=none, below of=s2, node distance=27pt] {$y= 1$};
\draw (s0) edge node{$y=2\cdot y\gets 7$} (s1);
\draw (s0) edge[ematrix] ++ (0,-0.75);
\draw (s1) edge[ematrix] ++ (0,-0.75);
\draw (s1) edge node{$z=2\cdot z\gets 4$}  (s2);
\draw (s2) edge[ematrix] ++ (0,-0.75);
\end{tikzpicture}
\end{center}

Here, the states are $q \equiv z=1\cdot y\gets3 +  z=2\cdot
z\gets4\cdot\dup\cdot y\gets1$ and $r \equiv y\gets 1$. 



\section{Bisimulation of \NetKAT Automata}

We are now ready to consider algorithms for checking whether two \NetKAT
automata are language-equivalent. When we show two \NetKAT automata have the accept the
same set of guarded strings, this means that the two associated \NetKAT
programs, as network policies, have the same behavior.

The tool we will use to decide equivalence of \NetKAT automata is a
kind of relation between sets of states of one automaton and sets of states of
another called a bisimulation. By using sets of states for the bisimulation, we
will be performing determinization ``on the fly'' (were we not to do this,
unfortunately automata could be language equivalent but \emph{not} bisimilar).
Accordingly, in the following developments, one may equivalently think of
$Q\subset S$ as a set of states of an automaton, or a state in the
determinization of the same automaton.

Formally, our bisimulation is a relation in $B \subseteq \Pk \times
\mathcal{P}(S) \times \mathcal{P}(T)$ subject to additional requirements.
When the relation $B$ is clear, for the tuple $(\pk, Q_S, Q_T) \in B$, we will
write:
\[ Q_S \approxp{\pk} Q_T \]

In addition, if we have that $(\pk, Q_S, Q_T)\in B$ for \emph{all} $\pk\in \Pk$,
then we write:
\[ Q_S \approx Q_T \]

\begin{definition}[Bisimulation]
    A \emph{bisimulation} between \NetKAT automata $\mathcal{A} = (S, s_0, \epsilon_S, \delta_T)$
    and $\mathcal{B} = (T, t_0, \epsilon_T, \delta_T)$ is a relation $B
    \subseteq \Pk \times \mathcal{P}(S) \times \mathcal{P}(T)$ such that:
\begin{itemize}[(a)]
    \item $\{s_0\} \approx \{t_0\}$,
    \item $Q_S \approxp{\pk} Q_T \Rightarrow \bigcup_{s\in Q_S} \epsilon_S~\pk~s
                                         = \bigcup_{t\in Q_T} \epsilon_T~\pk~t$,
    \item $Q_S \approxp{\pk} Q_T \Rightarrow
      (\bigcup_{s\in Q_S} \delta_S\ s\ \pk)\vert_{\pkp} \approxp{\pkp}
      (\bigcup_{t\in Q_T} \delta_T\ t\ \pk)\vert_{\pkp}$
\end{itemize}
\end{definition}

The definition above states that two automata are bisimilar (i.e. a bisimulation
exists between them) if (a) the singleton sets containing their start states are
bisimilar for all packets, (b) the value of the respective observation functions
are equal for bisimilar sets, and (c) transitions performing the same tests
and assignments from bisimilar state sets must also go to bisimilar state sets.

\begin{theorem}
    For two \NetKAT automata $\mathcal{A}$ and $\mathcal{B}$, there is a
    bisimulation $B$ between them if and only if $L(\mathcal{A}) = L(\mathcal{B}).$
\end{theorem}
\begin{proof}
    Suppose there is a bisimulation between $\mathcal{A}$ and $\mathcal{B}$.
    Then let $w = \pk_{in}\cdot \pk_1 \ddd \ldots \ddd \pk_n \ddd \pk_{out} \in
    L(\mathcal{A})$. If $n = 0$ (i.e., $w = \pk_{in} \cdot \pk_{out}$), we know
    $w\in L(\mathcal{B})$ by properties (a) and (b) of the bisimulation.     
    Let $P_S = \delta_S^\star\ s_0\ (\pk_{in} \cdot \pk_1 \ddd \ldots \ddd \pk_n)$ and let
        $P_T = \delta_T^\star\ s_0\ (\pk_{in} \cdot \pk_1 \ddd \ldots \ddd \pk_n)$.
    By applying property (b) inductively on $n$, we know that for each $\pk \in
    \Pk$, $P_S\vert_p \approxp{\pk} P_T\vert_p$. It follows by
    property (b) that $w\in L(\mathcal{B})$. The reverse direction is similar.
\end{proof}

\subsection{Computing Bisimulations}\label{sec:computing-bisim}

We present two methods for computing bisimulations on \NetKAT automata as
described above.

\subsubsection{Symbolic Packets}\label{sec:sp}
A natural way to compute a bisimulation of two automata is simply to start by
trying to relate the two start states and exploring outwards. This approach
can be seen as a modification of Hopcroft and Karp's standard algorithm for bisimulation
checking of finite automata \cite{hopcroft-karp1971}. That is, for each
pair of states that are related, we recursively ensure that the pairs of states
reachable by the same transformation to the symbolic packet are also related.

In other words, we have a function
$\bisim\colon \mathcal{P}(\Pk) \to \mathcal{P}(S) \to \mathcal{P}(T) \to 2$
which is defined as:
\[
\bisim\ P\ Q_S\ Q_T \triangleq
            \forall \pk\in P\colon (\bigcup_{s\in Q_S} \epsilon_1\ s\ \pk) =
                                   (\bigcup_{t\in Q_T} \epsilon_T\ t\ \pk) \wedge
            \bigwedge_{\substack{\pkp,Q_1', Q_T' \in \Pk \times  \mathcal{P}(S) \times \mathcal{P}(T)\colon \\
                        Q_S' = (\bigcup_{s\in Q_S} \delta_S\ s\ \pk)\vert_{\pkp} \wedge \\
                        Q_T' = (\bigcup_{t\in Q_T} \delta_T\ t\ \pk)\vert_{\pkp}}} \bisim\ \ \pkp\ \ Q_S'\ \ Q_T'
\]

The decision procedure for equivalence is given in \Cref{fig:sym-pack-alg}.

\begin{figure}
\begin{enumerate}
\item Start with an empty relation $B$.
\item \label{step:init} Let $P=\Pk, Q_S = \{s_0\}, Q_T = \{t_0\}$.
\item \label{step:check-eps} For the current $(P, Q_S, Q_T)$, check that the packets reachable by applying the
observation function on each side to this symbolic packet is the same set, i.e.
that $(\bigcup_{s\in Q_S} \epsilon_S\ s\ \pk) = (\bigcup_{t\in Q_T} \epsilon_T\ t\ \pk)$.
If they are, add $Q_S \approxp{\pk} Q_T$ to $B$. Otherwise, stop and return
failure.
\item \label{step:iter-delta} For each $\pkp\in\Pk$, compute:
\[
Q_S' = (\bigcup_{s\in Q_S} \delta_S\ s\ \pk)\vert_{\pkp}
\] and \[
Q_T' = (\bigcup_{t\in Q_T} \delta_T\ t\ \pk)\vert_{\pkp}
\]
and perform, as a recursive call, step~\ref{step:check-eps} with $\pk=\pkp,
Q_S=Q_S'$, and $Q_T=Q_T'$.
\item When all the recursive calls conclude, return $B$.
\end{enumerate}
\caption{Algorithm for computing bisimulation by Symbolic
Packets}\label{fig:sym-pack-alg}
\end{figure}.

% MM rewrite this --- can be made clearer with new pseudocode
%%e need not iterate all possible
%%successor packets because the FDD representation of transitions already keeps
%%them in equivalence classes, so we need only iterate over the leaves of the FDD
%%(see \Cref{sec:fdd}).

This method computes the minimum (by containment) bisimulation:

\begin{theorem}
If the algorithm in \Cref{fig:sym-pack-alg} succeeds, then for the relation $B$:
\begin{enumerate}
\item $B$ is a bisimulation.
\item For any bisimulation $B'$, we have $B \subseteq B'$.
\end{enumerate}
\end{theorem}
\begin{proof} \ \\
\mm{Maybe both parts can be a bit more explicit...}
\begin{enumerate}
\item Requirement (a) of bisimulations is met by step~\Cref{step:init}.
Requirements (b) and (c) are checked inductively by steps~\Cref{step:check-eps}
and \Cref{step:iter-delta}.
\item For each triple added by step~\Cref{step:check-eps}, failing to add the
triple violates a requirement of bisimulations, and hence must be in any
bisimulation.
\end{enumerate}
\end{proof}




\subsection{NetKAT's Carry-On Packets}

NetKAT is designed to largely adhere to standard presentations of
automata and coalgebras. But there is one aspect of NetKAT automata
that is somewhat unusual. Recall that the transition function has type
$\delta\in S \to S^{\Pk \times \Pk}$, which suggests that the
``characters'' consumed by the automaton are pairs of
packets. Operationally, these packets can be thought of as
input-output pairs---i.e., a switch matches on the input packet and
produces a possibly-modified output packet.

However, unlike the usual notion of acceptance for a deterministic
automaton, the acceptance function does not consume packets from the
garded string in pairs even though the alphabet of the automaton is
$\Pk\times \Pk$. In particular, when we take a transition
$\delta_{\pk\pk_1}(s)$ in \Cref{accept_transition}, the packet $\pk_1$
remains as a part of the string that is processed in the next
state. Hence, NetKAT automata exhibit a kind of \emph{carry-on state}
corresponding to the current value of the packet at each program
point. Interestingly, it turns out that these carry-on packets can be
understood as fundamental from the coalgebraic perspective. This
account of carry-on packets has not been given in prior work on NetKAT
and is essential for understanding the design of the new symbolic data
structures and bisimulation algorithms developed in this paper.

First, observe that the set of guarded strings $\pow\GS = \pow\GSdef$
is isomorphic to the set $\left(2^{\Pk\times\Pk}\right)^{\Pk^\star}$:
%
(i) $\pow X$ is in one-to-one correspondence to characteristic
functions on $X$ (i.e., $2^X$);
%
(ii) lists are isomorphic to
products;
%
(iii) we can remove $\dup$, which essentially is a one-element set and
hence neutral in a product;
%
(iv) we can curry the resulting function.
\begin{equation}\label{eq:iso}
\!\!\!\pow\GSdef \stackrel{\text{(i)}}\cong 2^{\GSdef} \stackrel{\text{(ii)}}\cong 2^{\Pk\times{(\Pk\times \dup)^\star}\times\Pk}  \stackrel{\text{(iii)}}\cong 2^{\Pk\times{\Pk^\star}\times\Pk} \stackrel{\text{(iv)}}\cong \left(2^{\Pk\times\Pk}\right)^{\Pk^\star}
\end{equation}
%
Hence, we can identify $\pow\GS$ with a set of the form $O^{A^\star}$,
for $O=2^{\Pk\times\Pk}$ and $A=\Pk$, which turns out to be a
well-known object: $O$-valued languages accepted by Moore coalgebras
$X \xrightarrow{\left<o,t\right>} O \times X^A$. In particular, there
is a unique map $\mathcal L$ assigning to each state $x\in X$ of a
Moore automaton the language it accepts:
\begin{equation}\label{eq:finalMoore}
\xymatrix@C=2cm{X \ar@{-->}[r]^{\mathcal L}\ar[d]_{\left<o,t\right>}& O^{A^*}\ar[d]_{\left<\varepsilon,\partial\right>}\\
O\times X^A \ar@{-->}[r]^{\mathsf{id}\times \mathcal L^A} & O\times (O^{A^*})^{A}
}
\end{equation}
Here, $\varepsilon$ and $\partial$ are language derivatives given by,
for $\xi\in A^{O^\star}$: $\varepsilon(\xi) = \xi(\lambda)$ and
$\partial(\xi)(a)(u) = \xi(au)$, where $\lambda$ denotes the empty
word. The map $\mathcal L$ can be ``read off'' from the above diagram,
%
\[
\mathcal L(s)(\lambda) = o(s) \quad\text{and} \quad \mathcal L(s)(au) = \mathcal L(t(s)(a))(u)
\]
%
and is the usual inductive definition of language acceptance in a
deterministic Moore automaton.

If we compare the type of a deterministic Moore automaton with the
type of a \NetKAT coalgebra, there is still a slight mismatch: $X \to
O \times X^A$ vs. $X \to O \times X^{A\times A}$ (for
$O=2^{\Pk\times\Pk}$ and $A=\Pk$). As it turns out the extra $A$ in
the alphabet of a \NetKAT coalgebra is what appears as carry-on packet
in the definition of $\accept$ and is acting as a kind of {\em
  semantic effect}.

If we care to eliminate this mismatch, we can construct a Moore
coalgebra from a \NetKAT algebra in which the carry-on bit is hidden
in the state space. Given a \NetKAT coalgebra $(S, \left<\epsilon,
\delta\right>)$ we build a Moore automaton $(S^\Pk,
\left<\epsilon^\sharp, \delta^\sharp\right>)$ where the states are
packet-indexed vectors, and both the observation and transition
functions process a carry-on packet:
%
\[
\epsilon^\sharp_{\pk\pkp}(\varphi) = \epsilon_{\pk\pkp}(\varphi(\pk)) \qquad\qquad
\delta^\sharp_{\pk\pkp}(\varphi) = \delta_{\pk\pkp}(\varphi(\pk))
\]
%
The $\accept$ function is then defined using the unique map as defined
in \Cref{eq:finalMoore} from the coalgebra $(S^\Pk,
\left<\epsilon^\sharp, \delta^\sharp\right>)$ to
$(O^{A^\star},\left<\varepsilon,\partial\right>)$. Note how when
instantiating $O=2^{\Pk\times\Pk}$ and $A=\Pk$ we have that
$O^{A^\star}=\left(2^{\Pk\times\Pk}\right)^{\Pk^\star}$ which as we
showed above is isomorphic to $\pow\GS$. Since $\accept$ operates over
elements of $S$ and $\mathcal L$ over $S^\Pk$ we use an injection
$\eta \colon S \to S^\Pk$ given by $\eta(s)(\pk) = s$.\footnote{The
injection $\eta$ is in fact a natural transformation between two
functors, the identity functor and $(-)^\Pk$. The latter is a monad,
the so-called reader monad, and $\eta$ is the unit of that monad.}
%
\[
\xymatrix@C=1.8cm{
X\ar[dr]_{\left<\epsilon, \delta\right>}\ar[r]_\eta\ar@/^{2.5ex}/[rr]^\accept&X^\Pk \ar[d]_{\left<\epsilon^\sharp, \delta^\sharp\right>}\ar@{-->}[r]_{\mathcal L} & \pow\GS\ar[d]^{\left<\overline\varepsilon,\overline\partial\right>}\\
&2^{\Pk \times \Pk}\times (X^{\Pk})^\Pk\ar@{-->}[r]& 2^{\Pk \times \Pk}\times \pow\GS^{\Pk}
}
\]
%
The functions $\overline\varepsilon$ and $\overline\partial$ apply
$\varepsilon$ and $\partial$ after the isomorphism map in
\Cref{eq:iso} yielding the following semantic derivative functions on
sets of guarded strings: $\overline\varepsilon (U) = \{\pk\cdot\pkp
\mid \pk\cdot\pkp\in U\}$ and $\overline\partial(U)(\pk) = \{\pkp\cdot
w\mid \pk\cdot\pkp\cdot \dup \cdot w\in U\}$. Instantiating the
definition of $\accept$ from the commutativity of the above diagram we
obtain:
%
\begin{align*}
\pk\cdot\pkp \in \accept(s)
&\iff \pk\cdot\pkp \in \mathcal L(\eta(s)) \\
&\iff \overline\varepsilon_{\pk\pkp}(\mathcal L(\eta(s)))=1\\
&\iff \epsilon^\sharp_{\pk\pkp}(\eta(s)) = 1\\
& \iff \epsilon_{\pk\pkp}(s) = 1\\
\pk \cdot \pkp \cdot \dup \cdot w \in \accept(s)
&\iff \pk \cdot \pkp \cdot \dup \cdot w \in \mathcal L(\eta(s)) \\
&\iff \pkp  \cdot w \in \overline\partial(\mathcal L(\eta(s)))(\pk) \\
&\iff \pkp \cdot w \in  \mathcal L(\delta^\sharp(\eta(s))(\pk))\\
&\iff \pkp \cdot w \in  \mathcal L(\delta(s)(\pk))\\
&\stackrel\dagger\iff \pkp \cdot w \in  \mathcal L(\eta(\delta_{\pk\pkp}(s)))\\
&\iff \pkp \cdot w \in \accept(\delta_{\pk\pkp}(s))
\end{align*}
In the step marked by $\dagger$ we use an auxiliary lemma
(\Cref{lem:stringvector}), which states the first packet in a string
determines the language of a vector of states:
\[
\forall \varphi\in X^\Pk\quad  \pk\cdot w \in \mathcal L ( \varphi) \iff  \pk\cdot w \in \mathcal L ( \eta(\varphi(\pk))).
\]

\paragraph*{Change in perspective: Weakest Preconditions}

Our last method for computing bisimulations combines ideas from the
previous two approaches. The focus is on computing what we call a
\emph{weakest precondition} for each pair of states. For a pair of
states $(s,t)$, the weakest precondition is the set of packets for
which $s$ and $t$ have the same behavior. Formally:
\begin{align*}
    &\wpre \in S \to T \to 2^\Pk \\
    &\wpre\ s\ t = \{ \pk \mid \pk\cdot w\in \accept(s) \Leftrightarrow
                             \pk\cdot w \in \accept(t),~\forall w\in \Pk\cdot(\dup\cdot\Pk)^\star\}
\end{align*}

It follows immediately from this definition that $\A$ and $\B$ are
equivalent if and only if $\wpre\ s_0\ t_0 = \Pk$. Our overall
strategy for the algorithm is to compute $wp\ s_0\ t_0$ by exploring
the reachable state space and computing weakest preconditions
incrementally as we discover pairs of states.

The details of our strategy follow from making two observations about
weakest preconditions. First, for any pair of states $(s,t)$,
%
\[ \wpre\ s\ t \subseteq \{ \pk \mid \epsA(s) = \epsB(t) \}. \]
%
Moreover if $(s,t)$ have no outgoing transitions, then the above statement is an
equality.
%
Second, whenever $(\sympk, (s,t)) \in \prevC\ (s',t')\ \sympkp$, then
\[ \sympkp \nsubseteq \wpre\ s'\ t' \Rightarrow \sympk \nsubseteq \wpre\ s\ t\]
%
This expression, in concrete packets, says that if $\delA(s) = s',
\delB(t) = t'$ and $\pkp \notin \wpre\ s'\ t'$, then we know $\pk
\notin \wpre\ s\ t$.

Thus for each pair $(s,t)$ that we discover, we initialize its weakest
precondition using the first observation, then iteratively remove
packets using the second observation and the current weakest
precondition for its successors.  This process means that whenever a
pair's weakest precondition is updated, we need to check all of its
\emph{predecessors} which may be affected. Ultimately, we compute a
fix point under these restrictions.

The connection to the forward algorithm of \Cref{fig:fwd-sym} is that
we perform a path-based traversal of the automaton from the start
states. The connection to the backward algorithm is that the weakest
preconditions are actually the complement of the set $B$ computed in
\Cref{fig:bkwd-alg}. The advantage over the backward algorithm is
avoiding unnecessary work by \emph{only} considering pairs of states
reachable by a path from the start states.

The Weakest Preconditions algorithm is shown in \Cref{fig:wp-alg}.

\begin{figure}
    \centering
    \begin{algorithmic}
        \State $R \gets \PkST; \todoR \gets \{(s_0, t_0)\}; \seenR \gets \emptyset$;
        \While{$\todoR$ is not empty $\wedge R_{s_0,t_0} = \Pk$}
            \State extract $(s, t)$ from $\todoR$;
            \If {$(s,t) \notin \seenR$}
                \State add $(s,t)$ to $\seenR$;
                \State $R_{s,t} \gets \{ \pk \mid \epsA(s) = \epsB(t) \}$;
                \State add all $\{(s',t') \mid (\sympkp, (s', t')) \in \nextC\ (s,t)\ (\Pk)\}$ to $\todoR$;
            \EndIf
            \State $\sympk \gets R_{s,t}$;
            \For {$(\sympkp, (s', t')) \in \nextC\ (s,t)\ (\Pk)$}
                \State $\sympk' \gets \{\sympk \mid (\sympk, (s,t)) \in (\prevC\ (s',t')\ (\sympkp - R_{s',t'}))\}$;
                \State $\sympk \gets \sympk - \sympk'$;
            \EndFor
            \If {$R_{s,t} \neq \sympk$}
                \State add all $\{(s',t') \mid (\sympkp, s',t') \in \prevC\ (s,t)\ (\Pk)\}$ to $\todoR$;
                \State $R_{s,t} \gets \sympk$;
            \EndIf
        \EndWhile
        \State \Return $R_{s_0,t_0} = \Pk$;
    \end{algorithmic}
    \caption{Weakest Precondition bisimulation.}\label{fig:wp-alg}
\end{figure}

\subsection{Correctness of Weakest Preconditions}

The next two results establish the correctness of the Weakest
Precondition algorithm.

\begin{lemma}\label{lem:wp-termination}
The algorithm in \Cref{fig:wp-alg} terminates.
\end{lemma}
\begin{proof}
Let $n = |R|$ (counting distinct concrete packets as different), $m = |S\times T - \seenR|$,
and $k = |\todoR|$. Then $(n,m,k)$ are finite and if $k$ reaches 0 then we terminate.
Thus it suffices to show that $(n,m,k)$ decreases lexicographically each
loop iteration. We argue using the paths as cases:
\begin{itemize}
    \item Suppose we enter the first if-statement ($(s,t)\notin \seenR$). Then
        $m$ decreases. For a given pair $(s,t)$ we can only execute this block
        once because it is added to $\seenR$ and nothing is removed from
        $\seenR$. So this is the first time we are modifying $R_{s,t}$, which
        means $R_{s,t} = \Pk$ before this. Thus $n$ is either unchanged or
        decreases. (It does not matter if we enter the second if-statement since
        $n$ can only decrease.)
    \item Suppose we do not enter the first if-statement, but we do enter the
        second if-statement. Observe that $|\sympk| \leq |R_{s,t}|$ since
        $\sympk$ is formed only by removing packets (starting from $R_{s,t}$).
        Moreover, we know that progress is made because we entered the
        if-statment: $\sympk \neq R_{s,t}$. Combining, we conclude that $|\sympk| <
        |R_{s,t}|$.  Thus by updating $R_{s,t}$ to $\sympk$, we know that
        $n$ decreases.
    \item Suppose we do not enter either if-statement. We know $n$ and $m$ are unchanged
        because the updates to $R$ and $\seenR$ only happen in the if-statements.
        $k$ decreases because we popped $\todoR$.
\end{itemize}
\end{proof}

\begin{restatable}[Correctness of Weakest Preconditions]{theorem}{wpcorrectness}\label{thm:wp-correct}
The algorithm in \Cref{fig:wp-alg} returns \true if there is a
bisimulation between $\mathcal{A}$ and $\mathcal{B}$ and returns
\false if there is no such bisimulation.
\end{restatable}


\subsection{Normal Form Representation of \NetKAT Expressions}\label{sec:nknf}

For each set of guarded strings, there are infinitely many \NetKAT
expressions expressing them.  Both the conversion to automata and
automata bisimulation algorithms benefit if some equivalences can be
detected using normal forms (we cannot hope to get all
equivalences---after all, this is the problem we set out to solve!).
To this end, we use a \NetKAT normal form \emph{NKNF} to represent all
\NetKAT expressions after parsing. The abstract syntax is given in
\cref{fig:nknf}.

A \emph{NKNF} expression is a disjunction of policy terms representing
a series of tests followed by a series of assignments (represented by
a single FDD, as shown in \Cref{sec:fdd}), interspersed by packet
duplication $\dup$ and by Kleene-starred \emph{NKNF} expressions.

We use a standard library set implementation for the disjunction,
which has the effect of normalizing for commutativity, associativity,
and idempotence (ACI) of the union operator. We use a list
representation for the concatenated terms, normalizing for
associativity of concatenation.

In addition to these data structure choices, we further normalize by
using smart constructors that apply certain \NetKAT axioms. In
particular we:
\begin{itemize}
  \item Reduce incompatable tests or incompatable assignment-then-tests to
  $\bot$, e.g. $x=3\cdot x=4 \equiv \bot$.
  \item Reduce sequences (i.e., concatenations) to $\bot$ whenever they contain
  $\bot$.
  \item Drop $\top$ from sequences unless it is the only element.
  \item Starred expressions are only ever starred once; i.e. ${p^\star}^\star$
    is always represented as $p^\star$.
\end{itemize}

\subsection{Path-based Optimization}

We have optimized our implementation of the backward algorithm to take
advantage of characteristic of certain queries. These optimizations
require tracking the relevance of each states with respect to the
initial query. While this is done by default in the forward algorithm,
this not required by the backward algorithm, and results in additional
computation during the discovery stage of the algorithm. As such, we
implemented them as options.

\subsubsection{Path Narrowing}

Some of the reachable states in the product automaton are not relevant
to the weakest precondition of the initial state, and as such can be
pruned. Our implementation of the backward algorithm supports path
narrowing as an option.

\subsubsection{Short Circuiting}

Short circuiting is an optimization which ends the bisimulation
algorithms as soon as enough information is discovered to be able to
decide the query. When deciding equivalence, this can be done as soon
as we find a packet history which is treated differently by the two
policies.

\begin{figure}
    \centering
    \begin{tabular}{c l l}
        \textbf{Syntax} & & \\
        \hline
        \text{Policies} $p$ $\Coloneqq$ & $f_1 = v_1 \dots f_n = v_n \cdot f_1 \gets v_1  \dots f_m \gets v_m$ & \emph{Tests-and-sets}             \\
                        & $\mid p_1 \cdot \dup \cdot p_2$ & \emph{Packet Duplication}                          \\
                        & $\mid p_1 \cdot e^\star \cdot p_2 $ & \emph{Kleene Star}                     \\
        \text{Expressions} $e$ $\Coloneqq$ & $p_1 + \dots + p_n$ & \emph{Disjunction}  \\
    \end{tabular}
    \caption{Normal form for NetKAT (NKNF)}
\label{fig:nknf}
\end{figure}


\subsection{On-the-Fly vs. Ahead-of-Time Automaton Construction}

A tradeoff that comes up frequently in automata-based analysis is
whether to compute the full relevant automata at the start and then
perform analysis on those or to compute individual states of automata
as they are demanded by the analysis. In the case of our bisimulation
algorithm, this on-the-fly approach means taking both Brzozowksi
derivatives and computing the product automaton in the main loop of
the particular bisimulation algorithm.

We have implemented both, and while there remains some possibility
that ahead of time construction allows one to amortize costs across
multiple queries in some way (say, if the specificaion automaton is
fixed), in our implementation we found that the on-the-fly approach
was the clear performant option (see \Cref{sec:eval}). The reason for
this seems to be mainly the pruning of the automaton in process (and
therefore saving work on parts of the automaton that need never be
constructed). It is worth noting that reachability of states in
\NetKAT automata is more subtle than in classic automata (in which
reachaibility coincides with graph reachability) because of the
carry-on packet.

% Forward Path during discovery
% Forward Provenance
\subsubsection{Forward Provenance}

When discovering states, we represent as a symbolic packet $p$ the
sets of packet that can get from the start state to the current
state. When generating counterexample path, each complete test
$\theta_{p} \in p$ corresponds to at least one feasible path between
the start state and this state:
\[
\mathit{h}_{0} \wedge \mathit{L} (\theta \wedge \eta \wedge \mathit{h}) \wedge \theta_{p}
\]
Where $\mathit{h}_{0}$ is the starting state, a list of triple
representing the hops in the path prefix, each with a test $\theta$, a
set of assignments $\eta$ and a resulting state $\mathit{h}$, and a
current path condition $\theta_{p}$.  This path is used as a prefix to
construct longer paths as we explore more states.
\begin{comment}
For example, if $\mathit{h}$ transitions to $\mathit{h'}$ on test $\theta'$ with assignments $\eta'$ and $\theta_{p} \cap \theta' = \theta'_{p} \not= \emptyset$, and the result of performing assignments $\eta'$ on packets in $\theta'_{p}$ results in packets matching $\theta_{c}$, we can construct a new path from $\mathit{h}_{0}$ to $\mathit{h'}$:
\[
\mathit{h}_{0} \wedge \mathit{L} (\theta \wedge \eta \wedge \mathit{h}) \cdot (\theta' \wedge \eta' \wedge \mathit{h'}) \wedge \theta_{c}
\]
\end{comment}
If we reach a state in the product automaton where the two expressions
are immediately distinguishable, we can produce a full path
demonstrating how a subset of the packets at the start state can reach
a state where they are not treated in the same way by the two
policies.

% TODO: example of forward provenance based on Fig 4
% MM: This example relies on figures that were cut; would need to rework
%%\paragraph*{Example} Consider the state $(q1,p1)$ from Figure~\ref{fig:ex-sp}. It is reached from state $(q0,p0)$ which are the start state of the joint automaton, either by matching $y=2 \wedge z=1$ and going through transition $y \gets7$, or by matching $y=2 \wedge z=2$ and going through transition $y\gets7$. As such, the forward provenance of state $(q1, p1)$ contains the two following provenance paths:
%%\[ (q0,p0),[(y=2 \wedge z=1, y\gets7, (q1,p1))], y=7 \wedge z = 1 \]
%%\[ (q0,p0),[(y=2 \wedge z=2, y\gets7, (q1,p1))], y=7 \wedge z = 2\]

%%Following the transition $y=7 \wedge z = 2 \cdot z\gets4$ from $(q1,p1)$ to $(q2,p1)$ results in the following provenance path for state $(q2,p1)$:
%%\[
%%(q0,p0),[(y=2 \wedge z=2, y\gets7, (q1,p1));
         %%(y=7 \wedge z=4, z\gets4, (q2,p1))], y=7 \wedge z = 4
%%\]


% Backward Path in the WP algorithm
% Backward provenance
\subsubsection{Backward Provenance}

When computing the weakest precondition for a state, we represent as a
symbolic packet the set of packets for which both policies have the
same behavior. Every $\bot$ leaf in the symbolic packet is the result
of either the behaviors of the policies being immediately different,
or being able to transition from that state to another state where the
behaviors are different. We keep track of the path between the current
state and the states with immediate difference by recording provenance
path of the form:
\[
\theta_{d} \wedge \mathit{L} (\mathit{h} \wedge \theta \wedge \eta ) \wedge \mathit{h}_{0}
\]

That is, packets in the current state $\mathit{h}$ matching test
$\theta_{d}$ can reach state $\mathit{h}_{0}$ by following the list of
hops from state $\mathit{h}$ following a transition with test $\theta$
and update $\eta$.
\begin{comment}
When computing the weakest precondition of a state $\mathit{h'}$ with
a transition $\theta'$ and $\eta'$ to $\mathit{h}$, we can use the
paths computed for $\mathit{h}$ as suffixes to construct longer paths
towards the start state. In particular, if the preimage of
$\theta_{d}$ under $\eta'$ results in $\theta'_{d}$, and $\theta'_{d}
\cap \theta' = \theta_{c} \not= \emptyset$, we can construct a new
path from $\mathit{h'}$ to $\mathit{h}_{0}$:
\[
\theta_{c} \wedge (\mathit{h'} \wedge \theta' \wedge \eta') \cdot \mathit{L} (\mathit{h} \wedge \theta \wedge \eta ) \wedge \mathit{h}_{0}
\]
\end{comment}
If a feasible path reaches the start state, we can use it to
demonstrate how packets are treated differently by the two policies as
a counterexample to their equivalence.

% TODO: maybe do an example that actually results in a backward path...
% MM: This example also depends on a deleted figure...
%%\paragraph*{Example} Consider the state $(q1, p1)$ from Figure~\ref{fig:ex-wp}. Assume that at this point of the algorithm, the backward provenance for state $(q2,p1)$ are paths
\[
  [y=7\wedge z\not=4, [],(q_2,p_1); y\not=7,[],(q_2,p_1)]
\]
That is, packets matching $(y=7\wedge z\not=4)\vee y\not=7$ are
treated differently by policies $q_2$ and $p_1$.  When computing the
weakest precondition of $(q_1,p_1)$ and consider the transition
$y=7\wedge z=2, z\gets4, (q_2,p_1)$, we would extend all backward
provenance paths for state $(q_2,p_1)$ matching $y=7\wedge z=4$, which
is none of them, which means that the contribution by that transition
to the backward provenance of state $(q_1,p_1)$ is empty: no packet
following that transition will be treated differently by $q_2$ and
$p_1$.

\section{Old Appendix: Weakest Precondition Correctness}

The next several results will establish the correctness of the Weakest
Preconditions algorithm (\Cref{fig:wp-alg}).
First, we need to formalize some terminology.
\[ \mathsf{successor}\ (s,t)\ (s',t') = \begin{cases}
    \true & \exists\sympkp\colon \sympkp\neq\emptyset \wedge (\sympkp, (s',t'))\in\nextC\ (s,t)\ (\Pk)\\
    \false & \text{otherwise}
\end{cases}
\]

\[ \mathsf{descendent}\ (s,t)\ (s',t') = \begin{cases}
    \true & \mathsf{successor}\ (s,t)\ (s',t')\\
    \true & \exists (s_i,t_i)\in S\times T\colon \mathsf{successor}\ (s,t)\ (s_i,t_i) \wedge \mathsf{descendent}\ (s_i,t_i)\ (s',t')\\
    \false & \text{otherwise}
\end{cases}
\]

At several places in the following proofs, we use the observation that if
$\pk\notin \wpre\ s\ t$, this is the same as saying there is a $w\in
\Pk\cdot(\dup\cdot\Pk)^\star$ such that $\pk\cdot w$ is in the symmetric
difference between $\accept(s)$ and $\accept(t)$, which we denote
$\accept(s) \triangle \accept(t)$.

\begin{lemma}\label{lem:one-step}
Suppose $R_{s,t} \subseteq \wpre\ s\ t$, and we reach the for-loop. Assume for
each pair of states $(s',t')$ in $\nextS\ (s,t)\ (\sympk)$, we already have
$\wpre\ s'\ t' = R_{s',t'}$. Then when we update $R_{s,t}$ at the end of this
iteration, we will have $\wpre\ s\ t = R_{s,t}$.
\end{lemma}
\begin{proof}
If $R_{s,t} = \wpre\ s\ t$ already, we are done, so assume $R_{s,t}\neq\wpre\ s\
t$. Then there is a $\pk \in \wpre\ s\ t - R_{s,t}$ and $w\in
(\dup\cdot\Pk)^\star$ for which $\pk\cdot \pkp \ddd w \in \accept(s)
\triangle \accept (t)$.  But this means there is a $(\sympk_i, \sympkp, s',t') \in
\nextS\ (s,t)\ (R_{s,t}$ such that $\pk\in\sympk_i$ and $\pkp\in\sympkp$,
because we know that $\wpre\ s'\ t' = R_{s',t'}$.
This $\sympk_i$ is removed from $R_{s,t}$ by the action of the for loop. We can
apply this argument to all packets remaining in $R_{s,t} - \wpre\ s\ t$.
\end{proof}

%%\begin{lemma}\label{lem:nextprev}
%%\mminline{add}
%%\end{lemma}
%%\begin{proof}
%%\end{proof}

\wpcorrectness*
\begin{proof}
We will show that $R_{s,t}$ approaches $\wpre\ s\ t$, which will justify the
correctness of the return value.

Now we establish an invariant for the while loop.
\[ \forall (s,t)\in S\times T\colon \wpre\ s\ t \subseteq R_{s,t} \tag{*}\label{wp-i1}\]
\begin{itemize}
    \item \emph{Initialization.} The first time we reach the top of the loop,
        $R = \PkST$ so \Cref{wp-i1} holds trivially.
    \item \emph{Maintenance.} Suppose \Cref{wp-i1} holds and we enter the loop
        ($\todoR$ is not empty). Then we need to show that the loop body
        maintains \Cref{wp-i1}. We look at the two places that we can subtract
        from $R_{s,t}$.
        \begin{itemize}
            \item In the first if-statement, we set $R_{s,t}$ to $\{\pk \mid
                \epsA(s) = \epsB(t)\}$. Then $\wpre\ s\ t \subset R_{s,t}$. To
                show this, we will show $\pk \notin R_{s,t}$ implies $\pk \notin
                \wpre\ s\ t$.
                Suppose $\pk \in R_{s,t}^C$. This means there is a packet $\pkp \in
                \epsA(s) \triangle \epsB(t)$, which in turn means $\pk\cdot\pkp \in
                \accept(s) \triangle \accept(t)$, and thus $\pk\notin \wpre\ s\
                t$, as needed.
            \item In the second if-statement, starting from $\sympk = R_{s,t}$
                intially, we update $\sympk$
                by subtracting some (possibly several) $\sympk'$.
                To show the loop invariant is maintained, we only
                need to show that each $\sympk' \cap \wpre\ s\ t = \emptyset$. This is
                true because for each $\sympk'$, we have, for a successor packet
                $\sympkp$ and state pair $(s',t')$,
                \[(\sympk',(s,t)) \in \prevC\ (s',t')\ (\sympkp). \]

                If $\sympk'$ is empty, then obviously $\sympk' \cap \wpre\ s\ t = \emptyset$.
                If $\sympk'$ is not empty, then it must be that $\sympkp - R_{s',t'}$ is not empty.
                By \Cref{wp-i1}, this means that $\sympkp - \wpre\ s'\ t' = \sympkp$.
                For each $\pkp\in \sympkp$, there is a packet % MM TODO: Lemma about prev to justify this
                $\pk'\in\sympk'$ such that $\delta_{\pk'\pkp}^{\A\times\B}(s,t) = (s',t')$.
                We know from $\sympkp \cap wp\ s'\ t' = \emptyset$, we have
                that $\pkp\cdot w \in \accept(s')\triangle \accept(t')$. Putting
                these things together,
                \[ \pk\ddd\pkp\cdot w \in \accept(s)\triangle \accept(t) \]
        \end{itemize}
\end{itemize}
If we return \false, then $R_{s_0,t_0} \neq \Pk$.
Using \Cref{wp-i1}, this means $\wpre\ s_0\ t_0 \neq \Pk$. By definition this
means there is a packet $\pk$ and $w\in \Pk\cdot(\dup\cdot\Pk)^\star$ for which
\begin{align*}
                &\pk\cdot w \in \accept(s_0) \triangle \accept(t_0)\\
    \Rightarrow &\pk\cdot w \in L(\A) \triangle L(\B)\\
    \Rightarrow & L(\A) \neq L(\B).
\end{align*}

All that remains is to show that if we conclude the loop with $R_{s_0,t_0} = \Pk$
(i.e. we return \true) then we have for all $(s,t)\in S\times T$,
\[R_{s,t} \subseteq \wpre\ s\ t\]
    Combining with \Cref{wp-i1}, this will mean that for all $(s,t)\in S\times T$,
\[R_{s,t} = \wpre\ s\ t,\]
which implies $L(\A) = L(\B)$ and concludes the proof.

We need one more invariant for the while loop.
\begin{align*}
\text{For each } (s,t) \in S\times T,\text{ at least one of the following holds:}\tag{**}\label{wp-i2}\\
    \parbox{0.6\textwidth}{
        \begin{enumerate}[(i)] % Item abels aren't working for some reason?
            \item $(s,t)\notin\seenR$
            \item $\wpre\ s\ t = R_{s,t}$
            \item $\exists (s',t')\in\todoR\colon \mathsf{descendent}\ (s,t)\ (s',t')$
            \item $(s,t) \in \todoR$
        \end{enumerate} }
\end{align*}

\begin{itemize}
    \item \emph{Initialization.} The invariant (\ref{wp-i2})(i) holds the first
    time reaching the loop since $\seenR$ is initially empty.
    \item \emph{Maintenance.} Suppose (\ref{wp-i2}) holds and we enter the loop
    body (meaning that $\todoR$ is not empty). We will show the loop body
    maintains the invariant.  Consider the $(s,t)$ that we remove from $\todoR$.
    For \emph{this} $(s,t)$, we know that it satisfies (\ref{wp-i2}). We
    consider in cases based on which item it satisfies:
    \begin{enumerate}[(i)]
        \item Suppose $(s,t)\notin\seenR$. In this case we add $(s,t)$ to $\seenR$ in the
        first if-statement. If it has successors, we add them to $\todoR$
        (establishing (iii)). If it does not, then that means
        $\wpre\ s\ t = \{ \pk \mid \epsA(s) = \epsB(t) \}$, which is exactly
        how we update $R_{s,t}$ (establishing (ii)).
%
        \item Suppose $\wpre\ s\ t = R_{s,t}$. Since we can only
        ever subtract from $R$, this is maintained by (\ref{wp-i1}).
%
        \item Suppose $(s,t)$ already has a descendent $(s',t')$ in $\todoR$. Since we are
            only removing $(s,t)$ from $\todoR$, its descendents remain on the
            worklist (maintaining (iii)) unless $(s,t)$ itself is the only
            descendent. Note that this means $(s,t)$ is its own successor. In
            this case, there are two subcases to consider, based on whether we
            enter the final if-statement.
        \begin{enumerate}
            \item If we enter the if-statement ($R_{s,t} \neq \sympk$), then we
            enqueue the predecessors of $(s,t)$, which includes $(s,t)$, maintaining
            the invariant (reestablishing (iii) and (iv)).
            \item Suppose we do not enter the if-statement. We show a
            contradition, revealing that this case is impossible. We may also assume
            $\wpre\ s\ t \neq R_{s,t}$ since otherwise we would have simply
            applied case (ii).
            So there is a $\pk\in R_{s,t} - \wpre\ s\ t$ and $w\in
            (\dup\cdot \Pk)^\star$ such that
            $\pk\cdot \pkp\cdot w \in \accept(s)\triangle\accept(t)$. We proceed
            by induction on $w$:
            \begin{itemize}
              \item  If $w = \varepsilon$, then $\pk\cdot\pkp\in
                \accept(s)\triangle\accept(t)$. This is impossible because when
                $(s,t)$ is added to $\seenR$, we initialized $R_{s,t}$ to $\{\pk
                \mid \epsA(s) = \epsB(t) \}$ and have only removed packets.
                  \item  If $w = \dup\cdot \pkp_2 \cdot w'$ for some $w' \in
                    (\dup\cdot\Pk)^\star$. But $w$ cannot take us into any other
                    states because (\ref{wp-i2}) implies that, for any
                    successors $(s',t')\in \seenR - \todoR$, we have $R_{s',t'}
                    = \wpre\ s'\ t'$ and thus \Cref{lem:one-step} applies.
                    So $\delA(s) = s$ and $\delB(t) = t$. But this in turn means
                    that $\pkp\in R_{s,t} - \wpre\ s\ t$. So we can appeal to
                    induction for $\pkp$ and $w'$, leading to contradiction.
            \end{itemize}
        \end{enumerate}
%
        \item Suppose $(s,t) \in \todoR$. Assume also that $(s,t)\in\seenR$,
        that $\wpre\ s\ t \neq R_{s,t}$ and that no descendents of $(s,t)$ are
        in $\todoR$, because otherwise then removing $(s,t)$ is fine and
        (\ref{wp-i2}) is maintained by the applicable case above. First we note
        that from the fact that none of the descendents of $(s,t)$ are in
        $\todoR$, this also means none of any descendents' descendents are on
        $\todoR$. We also know that the immediate successors of $(s,t)$
        \emph{were} on the $\todoR$ because $(s,t)$ is already in $\seenR$---so
        they are also in $\seenR$. Since (\ref{wp-i2}) holds at the
        top of the loop, this only leaves the possibility that for each
        immediate successor of $(s,t)$, call it $(s_i, t_i)$, we must have
        $\wpre\ s_i\ t_i = R_{s_i,t_i}$. Then by \Cref{lem:one-step}, the for loop
        identifies remaining packets in $R_{s,t} - \wpre\ s\ t$ and removes
        them before updating $R_{s,t} = \wpre\ s\ t$ and thus establishing (ii).
    \end{enumerate}
%
    Now we still need to consider the possibility that $(s,t)$ is the
    descendent of some other $(s_p, t_p) \neq (s,t)$. For this $(s_p, t_p)$, we
    may assume it has no other descendents in $\todoR$, that $(s_p,
    t_p)\notin\todoR$, that $\wpre\ s_p\ t_p \neq R_{s_p,t_p}$, and that
    $(s_p,t_p)\in\seenR$, since otherwise removing $(s,t)$ does not affect the
    invariant anyway. Note that this means that $(s,t)$ is the only descendent,
    and therefore an immediate successor of $(s_p,t_p)$. Again, we have two
    cases based on whether we enter the final if-statement.
    \begin{enumerate}
        \item If we do, then $(s_p, t_p)$ is enqueued because it is an immediate
        predecessor of $(s,t)$, reestablishing the loop invariant.
        \item The other case is impossible by the same argument as case
        \ref{self-loop} above.
    \end{enumerate}
\end{itemize}

At the end of the loop, $\todoR$ is empty. Applying this to (\ref{wp-i2}), we
find out that for every pair $(s,t)$, either $(s,t)\notin\seenR$ or $\wpre\ s\ t = R_{s,t}$.
Since we know $(s_0, t_0)$ is enqueued initially, it cannot still be unseen at
the end of the loop. So when we return \true, it is the case that $\wpre\ s_0\
t_0 = \Pk$, which as we noted earlier, means that $\accept(s_0) = \accept(t_0)$
and thus $L(\A) = L(\B)$ as required.

In fact, the relation $R$ is a bisimulation. We argue each item following the
definition of bisimulation, \Cref{def:bisim}.
\begin{enumerate}
    \item We know (a) holds from the fact that when we return true, $R_{s_0,t_0} = \Pk$.
    \item Let $(\pk, s, t)\in R$. Then both (b)(i) and (b)(ii) follow from the
    fact that we know $R_{s,t} = \wpre\ s\ t$:
        \begin{enumerate}
        \item Let $\pkp\in\Pk$. Then we know
        $\pk\cdot\pkp\in\accept(s)\Leftrightarrow\pk\cdot\pkp\in\accept(t)$,
        which means that $\epsA(s) = \epsB(t)$.
        \item Let $\pkp\in\Pk$ and let $(s',t')\in S\times T$ such that
        $\delA(s) = s'$ and $\delB(t) = t'$. We know
        $\pk\cdot\pkp\ddd w \in \accept(s) \Leftrightarrow\pk\cdot\pkp\ddd
        w\in\accept(t)$ for any $w\in \Pk\cdot(\dup\cdot\Pk)^\star$.
        Combining with the fact that $\delA(s) = s'$ and $\delB(t) = t'$, this
        also means that $\pkp\cdot w\in\accept(s') \Leftrightarrow \pkp\cdot
        w\in\accept(t')$, and therefore $\pkp\in \wpre\ s'\ t'$. Thus $\pkp\in R_{s',t'}$.
        \end{enumerate}
\end{enumerate}
\end{proof}

\section{Definition of Ternary conditional}\label{ap:extra_fdds}
{
\newcommand{\colorTEXT}{black}
\newcommand{\colorMATH}{black}
\newcommand{\colorSYNTAX}{black}

\newcommand{\underbracketWith}[2]{\underbracket{#2}_{#1}}

\newcommand{\daraisTEXT}[1]{{\color{\colorTEXT}\textnormal{#1}}}
\newcommand{\daraisMATH}[1]{{\color{\colorMATH}\ifmmode\mathnormal{#1}\else\(\mathnormal{#1}\)\fi}}
\newcommand{\daraisSTAX}[1]{{\color{\colorSYNTAX}\textnormal{\texttt{#1}}}}

\newcommand{\daraisModeRM}[1]{\ifmmode\operatorname{\mathrm{#1}}\else\textrm{#1}\fi}
\newcommand{\daraisModeIT}[1]{\ifmmode\operatorname{\mathit{#1}}\else\textit{#1}\fi}
\newcommand{\daraisModeBF}[1]{\ifmmode\operatorname{\mathbf{#1}}\else\textbf{#1}\fi}
\newcommand{\daraisModeTT}[1]{\ifmmode\operatorname{\mathtt{#1}}\else\texttt{#1}\fi}
\newcommand{\daraisModeSC}[1]{\ifmmode\operatorname{\textsc{#1}}\else\textsc{#1}\fi}
\newcommand{\daraisModeMM}[1]{\ifmmode#1\else\(#1\)\fi}

\begin{itemize}[label=\textbf{-},leftmargin=*]
\item  We write \daraisMATH{\{ \overline {x_{i}}\} } and \daraisMATH{\{ x_{1},\ldots ,x_{n}\} } to denote the same type of object: a set
   of elements in \daraisMATH{\wp (X)}. The same notational convention applies for \daraisMATH{\{ \overline {k_{i}\mapsto v_{i}}\} }
   and \daraisMATH{\{ k_{1}\mapsto v_{1},\ldots ,k_{n}\mapsto v_{n}\} }: a finite map from keys to values in \daraisMATH{K \Mapsto  V}.
\item  For finite maps \daraisMATH{{\daraisModeIT{kvs}} \in  K \Mapsto  V}: we write \daraisMATH{{\daraisModeIT{kvs}}[k\mapsto v]} for
   extending the mapping \daraisMATH{{\daraisModeIT{kvs}}} to map \daraisMATH{k} to \daraisMATH{v}, and
   as a consequence removing any existing mapping for \daraisMATH{k}, so
   \daraisMATH{{\daraisModeIT{kvs}}[k\mapsto v] = ({\daraisModeIT{kvs}} \setminus  \{ k\} ) \uplus  \{ k\mapsto v\} }. In contexts where \daraisMATH{k} is
   known to be in the domain of \daraisMATH{{\daraisModeIT{kvs}}} we write \daraisMATH{{\daraisModeIT{kvs}}[k]} for
   accessing the value mapped by key \daraisMATH{k}; in contexts where \daraisMATH{k} is not
   known to be in the domain, we write \daraisMATH{{\daraisModeIT{kvs}}[k] \mathrel{?} v} defined to be
   the value mapped to \daraisMATH{k} if in the domain, or \daraisMATH{v} (a default value)
   in the case \daraisMATH{k} is not in the domain.
\end{itemize}

{\renewcommand{\arraystretch}{1.5}
\begingroup\color{\colorMATH}\begin{gather*}\begin{array}{l
  } \underline{\hspace{0.66em}}=\underline{\hspace{0.66em}}\mathrel{?}\underline{\hspace{0.66em}}\mathrel{\diamond }\underline{\hspace{0.66em}} \in  K \times  V \times  {\daraisModeRM{FDD}}_{K,V}(X) \times  {\daraisModeRM{FDD}}_{K,V}(X) \rightarrow  {\daraisModeRM{FDD}}_{K,V}(X)
  \cr  k = v \mathrel{?} t^{L} \mathrel{\diamond } t^{R} \triangleq  
  \cr  \daraisSTAX{cases}
  \cr  \hspace*{1.00em} k < \lfloor t^{L}\rfloor \sqcap \lfloor t^{R}\rfloor  \implies  \left[k \mathrel{\overset{?}{=}}  \{ v\mapsto t^{L}\}  \mathrel{\diamond } t^{R}\right]
  \cr  \hspace*{1.00em} k = \lfloor t^{L}\rfloor  < \lfloor t^{R}\rfloor  \implies  \left[k \mathrel{\overset{?}{=}}  \{ v \mapsto  {\daraisModeIT{vts}}^{L}[v] \mathrel{?} t^{L \prime}\}  \mathrel{\diamond } t^{R}\right]
  % \cr  \hspace*{1.00em} k = \lfloor t^{R}\rfloor  < \lfloor t^{L}\rfloor  \implies  \left[k \mathrel{\overset{?}{=}}  \{ v \mapsto  t^{L}\}  \uplus  (vts^{R} \setminus  \{ v\} ) \mathrel{\diamond } t^{R \prime}\right]
  \cr  \hspace*{1.00em} k = \lfloor t^{R}\rfloor  < \lfloor t^{L}\rfloor  \implies  \left[k \mathrel{\overset{?}{=}}  vts^{R}[v\mapsto t^{L}] \mathrel{\diamond } t^{R \prime}\right]
  % \cr  \hspace*{1.00em} k = \lfloor t^{R}\rfloor  = \lceil t^{L}\rfloor  \implies  \left[k \mathrel{\overset{?}{=}}  \{ v \mapsto  {\daraisModeIT{vts}}^{L}[v] \mathrel{?} t^{L \prime}\}  \uplus  ({\daraisModeIT{vts}}^{R} \setminus  \{ v\} )\}  \mathrel{\diamond } t^{R \prime}\right]
  \cr  \hspace*{1.00em} k = \lfloor t^{R}\rfloor  = \lceil t^{L}\rfloor  \implies  \left[k \mathrel{\overset{?}{=}}  {\daraisModeIT{vts}}^{R}[v\mapsto {\daraisModeIT{vts}}^{L}[v] \mathrel{?} t^{L \prime}] \mathrel{\diamond } t^{R \prime}\right]
  \cr  \hspace*{1.00em} \lfloor t^{L}\rfloor  < k \sqcap  \lfloor t^{R}\rfloor  \implies  \left[k^{L} \mathrel{\overset{?}{=}}  \{ \overline {v_{i}^{L}\mapsto (k=v\mathrel{?}t_{i}^{L}\mathrel{\diamond }t^{R})}\}  \mathrel{\diamond } (k=v\mathrel{?}t^{L \prime}\mathrel{\diamond }t^{R}) \right]
  \cr  \hspace*{1.00em} \lfloor t^{R}\rfloor  < k \sqcap  \lfloor t^{L}\rfloor  \implies  \left[ k^{R} \mathrel{\overset{?}{=}}  \{ \overline {v_{i}^{R}\mapsto (k=v\mathrel{?}t^{L}\mathrel{\diamond }t_{i}^{R})}\}  \mathrel{\diamond } (k=v\mathrel{?}t^{L}\mathrel{\diamond }t^{R \prime}) \right] 
  \cr  \hspace*{1.00em} \lfloor t^{L}\rfloor  = \lfloor t^{R}\rfloor  < k \implies  \left[ k^{L} \mathrel{\overset{?}{=}}  
                             \left\{ \begin{array}{l
                               } v \mapsto  \langle {\daraisModeIT{vts}}^{L}[v] \mathrel{?} t^{L \prime},{\daraisModeIT{vts}}^{R}[v] \mathrel{?} t^{R \prime}\rangle  
                               \cr  \mathrel{|} v \in  {\daraisModeRM{dom}}({\daraisModeIT{vts}}^{L}) \cup  {\daraisModeRM{dom}}({\daraisModeIT{vts}}^{R})
                               \end{array}\right\} 
                            \mathrel{\diamond } (k=v\mathrel{?}t^{L \prime}\mathrel{\diamond }t^{R \prime}) 
                         \right]
  \cr  \daraisSTAX{where}
  \cr  \hspace*{1.00em} \begin{array}{lcl
       } \left[k^{L} \mathrel{\overset{?}{=}}  \underbracketWith{{\daraisModeIT{vts}}^{L}}{\{ \overline {v_{i}^{L}\mapsto t_{i}^{L}}\} } \mathrel{\diamond } t^{L \prime}\right] &{}={}& t^{L} \hspace*{1.00em}\daraisTEXT{{\daraisModeIT{when}}}\hspace*{1.00em} k \geq  \lfloor t^{L}\rfloor 
       \cr  \left[k^{R} \mathrel{\overset{?}{=}}  \underbracketWith{{\daraisModeIT{vts}}^{R}}{\{ \overline {v_{i}^{R}\mapsto t_{i}^{R}}\} } \mathrel{\diamond } t^{R \prime}\right] &{}={}& t^{R} \hspace*{1.00em}\daraisTEXT{{\daraisModeIT{when}}}\hspace*{1.00em} k \geq  \lfloor t^{R}\rfloor 
       \end{array}
  \end{array}
\end{gather*}\endgroup
}


}

\section{Operations on FDDs}\label{ap:binopfdd}

Though we define the syntax of $\FDD$ in terms of an $n$-ary
conditional, in later developments it will be convenient to also have
a conditional $f=n \mathbin ? t_1 \diamond t_2$. We present the formal
definition of this operation in \Cref{ap:extra_fdds}, which satisfies
the property:
\[
\fddSem{f=n \mathbin ? t_1 \diamond t_2} (\pk) =
\begin{cases}
\fddSem{t_1}(\sigma)&\text{if } \pk.f = n\\
\fddSem{t_2}(\sigma)&\text{if } \pk.f \neq n
\end{cases}
\]

Using this conditional, we can define a monadic composition operator
for \FDD s. Given an $\FDD[M]$ and a function $f \in M \to \FDD[N]$
we can construct an $\FDD[N]$ as follows:
\begin{align*}
- &\bind - \colon  \FDD[M] \to (M \to \FDD[N] ) \to \FDD[N] \\
m &\bind f \triangleq f(m) \\
(f \stackrel ? = \{ n_1\mapsto t_1,\ldots, n_k\mapsto t_k\} \diamond t) &\bind f \triangleq  f=n_1 \mathbin ? (t_1 \bind f) \diamond \left(\cdots \diamond  \left(f=n_k \mathbin ? (t_k \bind f) \diamond  (t \bind f) \right)\right)\\
\end{align*}
The above operation satisfies:
\[
\fddSem{t \bind f} (\pk) = f (\fddSem{t}(\pk))(\pk)
\]

The function $\binopFDD$ (known as \emph{apply} in BDDs) is defined as follows:
\begin{align*}
&\binopFDD\colon (A\to B\to C)\to \FDD[A] \to \FDD[B] \to \FDD[C] \\
&\binopFDD (f) (t_1) (t_2) \triangleq \\ &\begin{cases}
        f (a) (b)  & t_1 = a \in A, t_2 = b\in B\\
       \fddn{k_2}{v_2}{\binopFDD(f)(a)(t_2')}{\binopFDD(f)(a)(t_2'')}
                     & t_1 = a, t_2 = (\fddn{k_2}{v_2}{t_2'}{t_2''}) \\
       \fddn{k_1}{v_1}{\binopFDD(f)(t_1')(b)}{\binopFDD(f)(t_1'')(b)}
                     & t_1 = (\fddn{k_1}{v_1}{t_1'}{t_1''}), t_2 = b\in B \\
       \fddn{k_1}{v_1}{\binopFDD(f)(t_1')(t_2')}{\binopFDD(f)(t_1'')(t_2'')}
       & t_1 = (\fddn{k_1}{v_1}{t_1'}{t_1''}),\\
       & t_2 = (\fddn{k_2}{v_2}{t_2'}{t_2''}), k_1=k_1, v_1=v_2 \\
       {k_1 \stackrel ? = \overline{\{
           \substack
           {v_1\mapsto \binopFDD(f)(t_1')(t_2''),\\
            v_2\mapsto \binopFDD(f)(t_1'')(t_2')} \}} \diamond
       \binopFDD (f)(t_1'')(t_2'')}
       & t_1 = (\fddn{k_1}{v_1}{t_1'}{t_1''}),\\
       & t_2 = (\fddn{k_2}{v_2}{t_2'}{t_2''}), k_1=k_1, v_1\neq v_2 \\
       \fddn{k_1}{v_1}{\binopFDD(f)(t_1')(t_2)}{\binopFDD(f)(t_1'')(t_2)}& t_1 = (\fddn{k_1}{v_1}{t_1'}{t_1''}),\\
                     &  t_2 = (\fddn{k_2}{v_2}{t_2'}{t_2''}), k_1<k_2\\
       \fddn{k_2}{v_2}{\binopFDD(f)(t_1)(t_2')}{\binopFDD(f)(t_1)(t_2'')}& t_1 = (\fddn{k_1}{v_1}{t_1'}{t_1''}),\\
                     & t_2 = (\fddn{k_2}{v_2}{t_2'}{t_2'}), k_1>k_2\\
    \end{cases}
\end{align*}

\section{Definition of paths}\label{ap:paths}
We define $\paths$ by:
\begin{align*}
    &\paths\colon \FDD[Y] \to \mathcal{P}(\Paths \times Y) \\
    &\paths\ (t) = \begin{cases}
        \{(\{\}, y)\}, & t = y\in Y\\
        \{(\{(k_1 \mapsto \mathsf{inl } v_1)\}\uplus p, y) \mid (p,y)\in\paths\ t_1\}\\
        \qquad \cup\quad \{(\{(k_1 \mapsto \mathsf{inr } \{v_1\})\}\uplus p, y) \mid
        (p,y)\in\paths\ t_2 \},& t=\fddn{k_1}{v_1}{t_1}{t_2}
    \end{cases}
\end{align*}


\section{Definition of Update Operator}\label{ap:update-op}
The update operator is defined  by:
\mminline{I need to fix $\rhd^{-1}$ to adhere to the right type}
\begin{align*}
&- \rhd - \colon \FDD[2] \to (K\Mapsto V) \to \FDD[2]\\
&\sympk \rhd u = \begin{cases}
\fddn{k_1}{v_2}{\sympk' \rhd u'}{\false}
    & \sympk = \fddn{k_1}{v_1}{\sympk'}{\sympk''}, u = \{(k_1 \mapsto v_2)\}\uplus
    u'\\
%
\fddn{k_1}{v_1}{\sympk' \rhd u}{\sympk''\rhd u}
    & \sympk = \fddn{k_1}{v_1}{\sympk'}{\sympk''}, k_1 \notin u\\
%
\fddn{k_1}{v_1}{(\true \rhd u')}{\false}
    & \sympk = \true, u = \{(k_1\mapsto v_1)\} \uplus u'\\
\false & \sympk = \false
\end{cases}\\
    &- \rhd^{-1} - \colon \FDD[2] + 1 \to K \Mapsto V \to \FDD[2] + 1\\
    &\sympk \rhd^{-1} u = \begin{cases}
        \bot & \sympk = \bot\\
        \bot & \sympk = \true, u \neq\emptyset \\
        \bot & \sympk = \fddn{k_2}{v_2}{\sympk'}{\sympk''}, (k_1\mapsto v_2)\in u, k_1 < k_2\\
        \bot & \sympk = \fddn{k_1}{v_1}{\sympk'}{\sympk''}, (k_1\mapsto v_2)\in u, v_1 \neq v_2 \\
        \true & \sympk = \true, u=\emptyset \\
        \false & \sympk = \false\\
        (\sympk' \cup \sympk'') \rhd^{-1} u'
            & \sympk = \fddn{k_1}{v_1}{\sympk'}{\sympk''}, u = \{(k_1 \mapsto v_1)\} \uplus u'
    \end{cases}
\end{align*}

\section{Full Results}\label{app:full-results}

\begin{figure}[ht]
  \begin{center}
   \scalebox{0.8}{
    \begin{tabular}{ |c|c|c|c|c|c|c|c|c|c|}

      \hline
        \bfseries Topology & \bfseries Query & \bfseries Nodes & \bfseries Edges & \bfseries Size (KB) & \bfseries Fwd & \bfseries WP & \bfseries WPPP & \bfseries A-Comp & \bfseries A-Fwd %& \bfseries A-Bwd & \bfseries A-WP
        \csvreader{results.csv}{}
        {\\\hline \csvcoli & \csvcolii & \csvcoliii & \csvcoliv & \csvcolv & \csvcolvi & \csvcolvii & \csvcolviii & \csvcolix & \csvcolx }%\csvcolxi  \csvcolxii
        \\ \hline
    \end{tabular}}
    % MM: csvcolxi and csvcolxii were undefined for me so I removed them. Is
  % ther ea way to fix this?
  \end{center}
\caption{Evaluation Results}
\label{fig:eval-results}
\end{figure}

\begin{comment}
\begin{figure}[ht]
\begin{center}
  \scalebox{0.65}{
    \begin{tabular}{ |c|c|c|c|c|c|c|c|c|c|}
    \hline
    Topology & Query & \multicolumn{3}{|c|}{On-the-fly} & \multicolumn{5}{|c|}{Automata Compilation} \\ \hline
             &       & Fwd & WP & WP + Opt & Comp & Fwd & Bwd & WP & WP + Opt \\\hline
Airtel & reachability & 0.10 & 0.13 & 0.10 & 0.03 & 0.12 & 0.17 & 0.13 & 0.13 \\\hline
Airtel & slicing & 1.92 & 0.27 & 0.27 & 0.05 & 0.57 & 0.40 & 0.18 & 0.18 \\\hline
Airtel & unreachability & 0.37 & 0.12 & 0.11 & 0.03 & 0.22 & 0.12 & 0.12 & 0.12 \\\hline
Arpa & reachability & 0.32 & 0.57 & 0.30 & 0.26 & 0.48 & 1.05 & 0.49 & 0.49 \\\hline
Arpa & slicing & 46.63 & 1.68 & 2.31 & 0.44 & 7.31 & 5.61 & 0.89 & 0.90 \\\hline
Arpa & unreachability & 6.70 & 0.53 & 0.50 & 0.26 & 1.96 & 0.49 & 0.50 & 0.50 \\\hline
Belnet & reachability & 0.20 & 0.29 & 0.20 & 0.11 & 0.27 & 0.28 & 0.27 & 0.27 \\\hline
Belnet & slicing & 12.90 & 0.82 & 0.95 & 0.18 & 3.31 & 1.85 & 0.46 & 0.46 \\\hline
Belnet & unreachability & 1.61 & 0.30 & 0.25 & 0.11 & 0.68 & 0.27 & 0.28 & 0.28 \\\hline
Compuserv & reachability & 0.07 & 0.08 & 0.06 & 0.02 & 0.08 & 0.08 & 0.08 & 0.08 \\\hline
Compuserv & slicing & 1.01 & 0.16 & 0.16 & 0.03 & 0.31 & 0.23 & 0.11 & 0.11 \\\hline
Compuserv & unreachability & 0.21 & 0.08 & 0.07 & 0.02 & 0.13 & 0.08 & 0.08 & 0.08 \\\hline
Deltacom & reachability & 172.07 & 116.31 & 44.56 & 80.83 & 85.05 & 206.74 & 93.63 & 90.33 \\\hline
Deltacom & slicing & * & 461.14 & * & 124.14 & * & * & 190.65 & 169.52 \\\hline
Deltacom & unreachability & * & 106.36 & 103.63 & 79.62 & *  & 83.97 & 86.92 & 84.49 \\\hline
Layer42 & reachability & 0.01 & 0.02 & 0.01 & 0.00 & 0.02 & 0.02 & 0.02 & 0.02 \\\hline
Layer42 & slicing & 0.03 & 0.02 & 0.02 & 0.00 & 0.02 & 0.02 & 0.02 & 0.02 \\\hline
Layer42 & unreachability & 0.02 & 0.02 & 0.01 & 0.00 & 0.02 & 0.02 & 0.02 & 0.02 \\\hline
Missouri & reachability & 12.93 & 10.55 & 4.40 & 7.52 & 9.01 & 9.91 & 9.43 & 9.49 \\\hline
Missouri & slicing & * & 50.42 & 107.82 & 12.96 & * & * & 20.62 & 19.46 \\\hline
Missouri & unreachability & 641.43 & 10.59 & 10.03 & 7.61 & 61.16 & 9.09 & 9.19 & 9.21 \\\hline
Sanet & reachability & 1.91 & 2.34 & 0.98 & 1.23 & 1.79 & 3.91 & 1.98 & 2.00 \\\hline
Sanet & slicing & 501.94 & 7.36 & 13.19 & 2.09 & 63.35 & 155.44 & 3.59 & 3.60 \\\hline
Sanet & unreachability & 49.61 & 1.98 & 1.85 & 1.23 & 9.36 & 1.79 & 1.81 & 1.82 \\\hline
Shentel & reachability & 0.58 & 0.57 & 0.30 & 0.23 & 0.45 & 1.40 & 0.53 & 0.54 \\\hline
Shentel & slicing & 39.69 & 1.48 & 1.98 & 0.38 & 6.23 & 4.71 & 0.80 & 0.81 \\\hline
Shentel & unreachability & 6.45 & 0.48 & 0.48 & 0.23 & 2.02 & 0.44 & 0.45 & 0.46 \\\hline
Telcove & reachability & 9.57 & 14.09 & 5.54 & 9.77 & 11.55 & 12.76 & 11.64 & 11.67 \\\hline
Telcove & slicing & * & 61.01 & 143.48 & 16.29 & * & * & 24.97 & 23.92 \\\hline
Telcove & unreachability & 552.72 & 14.22 & 9.80 & 9.83 & 46.53 & 11.48 & 11.69 & 11.70 \\\hline
Uunet & reachability & 2.19 & 3.47 & 1.52 & 2.05 & 2.87 & 7.48 & 2.93 & 2.96 \\\hline
Uunet & slicing & 1088.79 & 12.63 & 24.20 & 3.43 & 142.75 & 91.90 & 5.88 & 5.89 \\\hline
Uunet & unreachability & 69.51 & 3.31 & 2.51 & 2.08 & 10.89 & 2.89 & 2.94 & 2.95 \\\hline
ft2 & reachability & 0.01 & 0.01 & 0.01 & 0.00 & 0.01 & 0.01 & 0.01 & 0.01 \\\hline
ft4 & reachability & 0.47 & 0.47 & 0.39 & 0.13 & 0.47 & 0.47 & 0.48 & 0.48  \\\hline
ft4 & slicing & 18.43 & 1.05 & 1.43 & 0.23 & 3.06 & 2.36 & 0.69 & 0.69 \\\hline
ft6 & reachability & 19.80 & 16.60 & 8.12 & 11.22 & 14.71 & 15.70 & 15.10 & 15.12 \\\hline
ft6 & slicing & * & 68.28 & 210.98 & 19.86 & * & *  & 31.76 & 30.86 \\\hline
ft6 & unreachability & * & 57.70 & 49.22 & 45.56 & 273.05 & 48.48 & 50.64 & 48.28 \\\hline
ft8 & reachability & 642.28 & 474.65 & 217.40 & 391.13 & 404.44 & *  & 435.58 & 417.83 \\\hline
  \end{tabular}}
\end{center}
\caption{Evaluation Results}
\label{fig:eval-results}
\end{figure}
\end{comment}


\bibliographystyle{ACM-Reference-Format}
\bibliography{refs}
\end{document}
