\appendix

\section{Product Automata}
\label{app:product}

To simplify the traversal of two automata simultaneously, some of our
bisimulation algorithms use a product construction analogous to the
one for classic finite automata. For two \NetKAT automata $\mathcal{A} = (S, s_0, \epsilon^\A, \delta^\A)$ and $\mathcal{B}
= (T, t_0, \epsilon^\B, \delta^\B)$, we define the product automaton by $\A\times\B =
(S\times T, (s_0, t_0), \epsilon^{\A\times\B}, \delta^{\A\times\B})$, where:
\begin{align*}
  &\epsilon^{\A\times\B}\colon S\times T \to 2^{\Pk\times\Pk}\\
  &\epsilon^{\A\times\B}(s,t)(\pk,\pkp) = (\epsilon^\A(s)(\pk,\pkp) = \epsilon^\B(t)(\pk,\pkp))
\end{align*}
and
\begin{align*}
  &\delta^{\A\times\B}\colon S\times T \to (S\times T)^{\Pk\times\Pk}\\
  &\delta^{\A\times\B}(s,t)(\pk,\pkp) = (\delta^\A(s)(\pk,\pkp), \delta^\B(t)(\pk,\pkp))
\end{align*}

\section{Proofs}

\begin{lemma}\label{lem:stringvector}
\[\forall \varphi\in X^\Pk\quad  \pk\cdot w \in \mathcal L ( \varphi) \iff  \pk\cdot w \in \mathcal L ( \eta(\varphi(\pk)))\]
\end{lemma}
\begin{proof}
Let $\varphi\in X^\Pk$. We can reason as follows, by case analysis on $w$.

\medskip
\noindent\fbox{$w = \pkp$}
\begin{align*}
&\pk\cdot \pkp \in \mathcal L ( \varphi)\\
%
\iff &\overline\varepsilon_{\pk \pkp} (\mathcal L ( \varphi))=1\\
\iff &\epsilon^\sharp_{\pk \pkp} ( \varphi)=1\\
\iff &\epsilon_{\pk \pkp} ( \varphi(\pk))=1\\
\iff &\epsilon^\sharp_{\pk \pkp} ( \eta(\varphi(\pk)))=1\\
\iff &\overline\varepsilon_{\pk \pkp} (\mathcal L  (\eta(\varphi(\pk)))=1\\
\iff  & \pk\cdot \pkp  \in \mathcal L ( \eta(\varphi(\pk)))\\
\end{align*}

\medskip
\noindent\fbox{$w = \pk_1 \cdot w_1$}
\begin{align*}
&\pk\cdot \pk_1 \cdot w_1 \in \mathcal L ( \varphi)\\
%
\iff & \pk_1 \cdot w_1 \in \overline\partial_{\pk} (\mathcal L ( \varphi))\\
\iff &\pk_1 \cdot w_1 \in \mathcal L ( \delta^\sharp_{\pk} (\varphi))\\
\iff &\pk_1 \cdot w_1 \in \mathcal L ( \lambda\pkp. \delta^\sharp_{\pk\pkp} (\varphi))\\
\iff &\pk_1 \cdot w_1 \in \mathcal L ( \lambda\pkp. \delta_{\pk\pkp} (\varphi(\pk)))\\
\iff & \pk_1 \cdot w_1 \in \mathcal L ( \lambda \pkp. \delta^\sharp_{\pk \pkp}  (\eta(\varphi(\pk))))\\
\iff & \pk_1 \cdot w_1 \in \mathcal L ( \delta^\sharp_{\pk}  (\eta(\varphi(\pk))))\\
\iff & \pk_1 \cdot w_1 \in \overline\partial_{\pk} (\mathcal L  (\eta(\varphi(\pk))))\\
\iff  & \pk\cdot  \pk_1 \cdot w_1  \in \mathcal L ( \eta(\varphi(\pk)))\\
\end{align*}
\end{proof}




\bisimnaivetermination*
\begin{proof}
  Let $n = |\Pk\times S\times T - R|$ and $m = \textsf{len}(\todoR)$. To prove termination we show
  $(n, m)$ decreases lexicographically every iteration of the while loop (and
  note that $m$ and $n$ are both always finite and nonnegative). Each loop iteration there are
  two cases.
  If $(\pk, s, t)\in R$, then $n$ is unchanged, but $m$ decreases because
  we extract a triple from $\todoR$ and do not execute the if-statement body. On the other
  hand if $(\pk, s, t)\notin R$ then $n$ decreases because $(\pk, s, t)$ is
  added to $R$.
  Termination follows.

  We call a a triple $(\pk, s, t) \in \Pk\times S\times T$ \emph essential if
  for any bisimulation $B$ between $\mathcal A$ and $\mathcal B$, we have $(\pk, s, t) \in B$.

  For the correctness of the return values, we need a loop invariant. Every
  iteration at the top of the loop, we have that (a) every triple in both $R$ and $\todoR$
  is essential, and (b) all the triples required by (b)(ii) of \Cref{def:bisim} for triples in $R$
  are in $\todoR$.
  The invariant holds intially because $R$ is empty and the triples in $\todoR$ are all
  required by (a) of \Cref{def:bisim}.
  Inside the loop we need only justify insertions.
  The insertion to $R$ is valid directly from the previous instance of the loop invariant because
  the triple being added was already in $\todoR$. The insertion to $\todoR$ maintains both parts of
  the invariant because of the previous invariant and the triples added are precisely those
  required by (b)(ii) of \Cref{def:bisim}.

  At last we see that if we return \false, then no bisimulation can exist because the current triple
  $(\pk, s, t)$ is essential (by the loop invariant) and thus the requirement (b)(i) of \Cref{def:bisim}
  is needed for $s,t$ but is violated.

  On the other hand if we return \true, then the fact that $R$ is a bisimulation
  follows from the last iteration of the loop invariant and the fact that
  $\todoR$ is empty.
\end{proof}

\soundcompletenaive*
\begin{proof}
Suppose $R$ is a bisimulation between $\A$ and $\B$.

\emph{Claim.} Let $(\pk, s, t)\in R$, and let $w\in  (\Pk\cdot\dup)^\star\cdot\Pk$.
Then $\pk\cdot w\in\accept(s) \Leftrightarrow \pk\cdot w\in\accept(t)$.

\emph{Proof of claim.} We proceed by induction on the length of $w$. For the base
case, let $w = \pkp$. Then from $(\pk, s, t)\in R$, we know by (b)(i) of
the definition of bisimulation that $\epsA(s) = \epsB(t)$. This means
$\pkp\in\epsA(s)\Leftrightarrow\pkp\in\epsB(t)$ and thus $\pk\cdot\pkp\in\accept(s)\Leftrightarrow
\pk\cdot\pkp\in\accept(t)$. For the inductive case, let $w = \pk_1\cdot\dup\cdot w'$. From
$(\pk, s, t)\in R$ and (b)(ii) of the definition of bisimulation, we know that
$(\pk_1, \delta_{\pk{\pk_1}}^{\A}(s), \delta_{\pk{\pk_1}}^{\B}(t))\in R$. We
apply the induction hypthosis to get
$\pk_1\cdot w'\in\accept(\delta_{\pk{\pk_1}}^{\A}(s))\Leftrightarrow \pk_1\cdot w'\in\accept(\delta_{\pk{\pk_1}}^{\B}(t)).$
Using the definition of accept, this means that
\[\pk\ddd{\pk_1}\cdot w'\in\accept(s)\Leftrightarrow \pk\ddd{\pk_1}\cdot w'\in\accept(t).\]
This is the same as
\[\pk\cdot w\in\accept(s) \Leftrightarrow \pk\cdot w\in\accept(t),\]
proving the claim.

The claim is enough to show $L(\A) = L(\B)$:\\
By (a) of the definition, $(\pk, s_0, t_0)\in R$.
Therefore by the claim $w\in \accept(s_0) \Leftrightarrow w\in\accept(t_0)$. By definition
this means $w\in L(\A) \Leftrightarrow w\in L(\B)$, and thus that $L(\A) = L(\B)$.

Now suppose on the other hand that $L(\A) = L(\B)$. Let
\[ R = \{ (\pk, s, t)\mid \pk\cdot w\in\accept(s)\Leftrightarrow\pk\cdot w\in\accept(t) \quad\forall w\in(\Pk\cdot\dup)^\star\cdot\Pk \}.\]
We will show $R$ is a bisimulation for $\A$ and $\B$, completing the proof.

Let $\pk\in\Pk$. For (a), we need to show that $(\pk, s_0, t_0) \in R$:
\begin{align*}
      & L(A) = L(B) \\
\iff  & \accept(s_0) = \accept(t_0) \\
\iff & \pk\cdot w\in\accept(s_0) \Leftrightarrow \pk\cdot w\in\accept(t_0)\\
\Rightarrow & (\pk, s_0, t_0) \in R
\end{align*}

Now let $(\pk, s, t)\in R$.
For (b)(i):
\begin{align*}
            & \pk\cdot w \in\accept(s) \Leftrightarrow \pk\cdot w\in\accept(t) &\forall w\in(\Pk\cdot\dup)^\star\cdot\Pk\\
\Rightarrow & \pk\cdot\pkp \in\accept(s) \Leftrightarrow \pk\cdot\pkp\in\accept(t) &\forall \pkp\in\Pk\\
\iff        & \epsA(s) = \epsB(t).
\end{align*}

Finally for (b)(ii):
\begin{align*}
            & \pk\cdot w \in\accept(s) \Leftrightarrow \pk\cdot w\in\accept(t) &\forall w\in(\Pk\cdot\dup)^\star\cdot\Pk\\
\Rightarrow & \pk\cdot\pk_1\cdot\dup\cdot w \in\accept(s) \Leftrightarrow \pk\cdot
\pk_1\cdot\dup\cdot w\in\accept(t) &\forall \pk_1\in\Pk, w\in(\Pk\cdot\dup)^\star\cdot\Pk\\
\iff        & (\pk_1, \delta_{\pk{\pk_1}}^{\A}(s), \delta_{\pk{\pk_1}}^{\B}(t))\in R &\forall \pk_1\in\Pk\\
\end{align*}

\end{proof}

\automataexpressions*
\begin{proof}
\asinline{add}
\end{proof}

\bkwdcorrect*
\begin{proof}
First we prove an invariant of the while loop that says that $B \cup \todoR$ must contain
all triples reachable by taking one step \emph{backwards} in each automaton from
triples in $B$. Formally:
\[ \delA(s) = s'\wedge \delB(t)=t' \Rightarrow (\pk, s, t)\in B \cup \todoR\qquad\forall
    (\pk, s, t)\in\PkST, \forall (\pkp, s', t')\in B
    \tag{*}\label{bkwd-i1} \]
is an invariant of the while loop.
\begin{itemize}
\item \emph{Initialization.} Since the first time we arrive at the top of the
while loop we have $B = \emptyset$, then \Cref{bkwd-i1} holds trivially.
\item \emph{Maintenance.} Now suppose \Cref{bkwd-i1} holds at the top of the loop,
and we enter the loop by extracting $(\sympkp, s', t')$ from $\todoR$.
If $(\sympkp, s', t')\in B$ already, nothing is modified, but then in that case
the loop invariant still holds because it is unaffected by removing
$(\sympkp,s',t')$ from $\todoR$. Otherwise we will add $(\sympkp, s',t')$ to $B$.
In that case, the triples found by
the for-loop, namely $(\pk, s, t)$ such that $\delA(s) = s'$ and $\delB(t) = t'$, are
precisely those needed to establish that the invariant still holds, and they are
added to $\todoR$, as needed.
\end{itemize}

Now suppose $\A$ and $\B$ are equivalent.
Let $R\subseteq\PkST$ be any bisimulation for $\A$ and $\B$.
We now demonstrate that under this condition, we have that
\[ B \cap R = \emptyset \tag{**}\label{bkwd-i2} \]
is an invariant of the while loop:
\begin{itemize}
\item \emph{Initialization.} \Cref{bkwd-i2} is true at the first time reaching the top
of the while loop. This is because, in the initialization for loop,
we can only add triples $(\pk, s, t)$ to $B$ when $\epsA(s) \neq \epsB(t)$,
which disqualifies them from being in $R$ by criterion (b)(i) of the definition
of bisimulation.
\item \emph{Maintenance.} \Cref{bkwd-i2} is maintained by the body of the while loop.
Suppose \Cref{bkwd-i2} is true at the top of loop, and suppose $\todoR$ is not
empty, so we enter the loop and extract $(\sympk, s, t)$ from $\todoR$.
Since without modification, the loop invariant will still hold, we need only
show that for each $(\pk, s, t)$ added to $B$ that we have $(\pk, s, t)\notin
R$. Based on the for loop, what we know about each $(\pk, s, t)$ added to $B$,
it is the case that $\delA(s) = s'$ and $\delB(t) = t'$ for some $(\pkp,s',t')
\in B$. But by \Cref{bkwd-i2}, we know $(\pkp,s',t')\notin R$. Indeed, if it
were the case that $(\pk,s,t)\in R$, then by criterion (b)(ii) of the
definition, we would have $(\pkp,s',t')\in R$. From this we conclude $(\pk, s,
t)\notin R$, and therefore \Cref{bkwd-i2} is maintained.
\end{itemize}

Consider what happens at the return statement. If we return \false, then
we know that $B_{s_0,t_0} \neq \emptyset$. Now suppose (for contradiction) there were
a bisimulation $R$ for $\A$ and $\B$. By criterion (a) of the
definition of bisimulation, $(\pk, s_0, t_0)\in R$ for any bisimulation $R$.
Thus there is some $\pk$ such that $(\pk, s_0, t_0)\in B\cap R$. But we showed
that by \Cref{bkwd-i2}, $B\cap R = \emptyset$. So $R$ does not exist, as needed.

On the other hand, if we return \true, then we
know that $B_{s_0,t_0} = \emptyset$. In this case we claim $B^C$ is a
bisimulation for $\A$ and $\B$. By the previous statement, it satisfies
requirement (a). In the initialization for-loop, we add every triple $(\pk,
s,t)$ to $B$ for which $\epsA(s) \neq \epsB(t)$. As a result we can conclude
that for every triple $(\pk, s, t)\in B^C$, we have $\epsA(s) = \epsB(t)$, as
needed for (b)(i). From the termination of the while loop, we have that $\todoR
= \emptyset$. Combining this with \Cref{bkwd-i1}, we have that for all $(\pkp, s',
t')\in B$, then it must be that whenever $\delA(s) = s'$ and $\delB(t) = t'$, we
have that $(\pk, s, t) \in B$. Well suppose some $(\pk, s, t)\notin B$ (i.e.,
$(\pk, s, t)\in B^C$), and suppose both $\delA(s) = s'$ then  $\delA(t) = t'$.
By the reasoning above, $(\pkp, s', t')$ cannot be in $B$ or else $(\pk, s, t)$
would have to be. So $B^C$ satisfies criterion (b)(ii), and therefore is a
bisimulation as required.
\end{proof}