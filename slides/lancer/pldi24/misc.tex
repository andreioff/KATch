\begin{figure}
  \begin{minipage}[t]{0.3\textwidth}
  \begin{align*}
      v &::= 0 \mid 1 \mid \ldots \mid n &\text{Values}\\
      f &::= f_1 \mid \ldots \mid f_k &\text{Fields}\\
      \pk &::= \{f_1 = v_1, \ldots, f_k = v_k\} &\text{Packets}\\
      p,q &::= & \text{Programs}
           \\&\mid\quad \top & \text{True}
           \\&\mid\quad \bot & \text{False}
           \\&\mid\quad f = v & \text{Test equals}
           \\&\mid\quad f \neq v & \text{Test not equals}
           \\&\mid\quad f \gets v & \text{Modification}
           \\&\mid\quad \text{dup} & \text{Duplication}
           \\&\mid\quad p + q & \text{Union}
           \\&\mid\quad p \cdot q & \text{Sequencing}
           \\&\mid\quad p^\star & \text{Iteration}
  \end{align*}
  \end{minipage}
  \begin{minipage}[t]{0.4\textwidth}
  \begin{align*}
      \Sem{-} \colon \Exp &\to \Pk \to \pow{\Pk^\star}\\[2mm]
      \Sem{\top}(\alpha) & \triangleq \{ \alpha \}\\
      \Sem{\bot}(\alpha) & \triangleq \emptyset\\
      \Sem{f \test v}(\alpha) & \triangleq \{ \alpha \mid \alpha.f = v \}\\
      \Sem{f \neq v}(\alpha) & \triangleq \{ \alpha \mid \alpha.f \neq v \}\\
      \Sem{f \mut v}(\alpha) & \triangleq \{ \alpha[v/f] \}\\
      \Sem{\text{dup}}(\alpha) & \triangleq \{ \alpha\alpha \}\\
      \Sem{p+q}(\alpha) & \triangleq \Sem{p}(\alpha) \cup \Sem{q}(\alpha)\\
      \Sem{p\cdot q}(\alpha) & \triangleq \{ \beta\gamma \mid \beta \in \Sem{p}(\alpha),\ \gamma \in \Sem{q}(\beta_{|\beta|}) \}\\
      \Sem{p^\star} & \triangleq \bigcup\limits_{n\geq 0} \Sem{p}^n\\
  \end{align*}
  \end{minipage}
  \caption{NetKAT syntax and semantics.}
\end{figure}

\begin{figure}
  \renewcommand{\arraystretch}{1.1}
  \begin{tabular}{lcl}
      % Syntax (left), Description (middle), and Semantics (right)
      $v ::= 0 \mid 1 \mid \ldots \mid n$ & Values & \\
      $f ::= f_1 \mid \ldots \mid f_k$ & Fields & \\
      $\pk ::= \{f_1 = v_1, \ldots, f_k = v_k\}$ & Packets & \\
      \hline
      Syntax & Description & Semantics \\
      \hline
      $p,q ::= $ & Programs & $(\alpha : \Pk) \to \pow{\Pk^\star}$ \\
      \qquad$\mid\quad \top$ & True & $\Sem{\top}(\alpha) \triangleq \{ \alpha \}$ \\
      \qquad$\mid\quad \bot$ & False & $\Sem{\bot}(\alpha) \triangleq \emptyset$ \\
      \qquad$\mid\quad f = v$ & Test equals & $\Sem{f \test v}(\alpha) \triangleq \{ \alpha \mid \alpha.f = v \}$ \\
      \qquad$\mid\quad f \neq v$ & Test not equals & $\Sem{f \neq v}(\alpha) \triangleq \{ \alpha \mid \alpha.f \neq v \}$ \\
      \qquad$\mid\quad f \gets v$ & Modification & $\Sem{f \mut v}(\alpha) \triangleq \{ \alpha[v/f] \}$ \\
      \qquad$\mid\quad \text{dup}$ & Duplication & $\Sem{\text{dup}}(\alpha) \triangleq \{ \alpha\alpha \}$ \\
      \qquad$\mid\quad p + q$ & Union & $\Sem{p+q}(\alpha) \triangleq \Sem{p}(\alpha) \cup \Sem{q}(\alpha)$ \\
      \qquad$\mid\quad p \cdot q$ & Sequencing & $\Sem{p\cdot q}(\alpha) \triangleq \{ \beta\gamma \mid \beta \in \Sem{p}(\alpha),\ \gamma \in \Sem{q}(\beta_{|\beta|}) \}$ \\
      \qquad$\mid\quad p^\star$ & Iteration & $\Sem{p^\star} \triangleq \bigcup\limits_{n\geq 0} \Sem{p}^n$ \\
  \end{tabular}
  \renewcommand{\arraystretch}{1.0}
\end{figure}

% \begin{figure}\small
%     \centering
%     \begin{tabular}{@{}lrcl@{}l|l}
%         \textbf{Syntax} & & &&& \textbf{Semantics: $\Sem{-} \colon \Exp \to \pow\GS$ }\\
%         \hline
%         \text{Values} &  $V \ni v$&$\Coloneqq$  &$0 \mid \ldots \mid n$ & & \hfill $\GS \triangleq \Pk\cdot(\Pk\cdot\text{dup})^\star\cdot\Pk$  \\
%         \text{Fields} & $F \ni f$&$\Coloneqq$& $f_1 \mid \ldots \mid$ & $f_k$ &
%         % GS semantics
%         \multirow{6}{*}{$
%         \begin{array}{lcl}
%         \Sem{a} &\triangleq& \{\pk \cdot \pk \mid \pk\in \pSem a \}\\
%         \Sem{f\gets v} &\triangleq& \{\pk \cdot \pkp \mid \pkp=\pk[v/f],\ \pk \in \Pk\}\\
%         \Sem{p + q} &\triangleq& \Sem{p} \cup \Sem{q}\\
%         \Sem{p\cdot q} &\triangleq& \Sem{p} \diamond \Sem{q}\\
%         \Sem{\dup} &\triangleq& \{ \pk\cdot \pk_1 \cdot \dup \cdot \pk_1 \mid \pk,\pk_1 \in \Pk\}\\
%         \Sem{p^\star} &\triangleq& \bigcup\limits_{n\geq 0} \Sem{p^n}
%         \end{array}
%         $}   \\
%         \text{Packets} &  $\Pk\ni \pk$&$\Coloneqq$&  $\{f_1 = n_1,$ & $\ldots,f_k = n_k\}$ &  \\
%         &&&&&  \\
%         %%\text{Histories}  & $h \Coloneqq pk::\langle\rangle \mid pk::h$ & & $\llbracket 1 \rrbracket h \triangleq \{h\}$ \\
%         $\text{Predicates}$ & $\Pred \ni a,b$&$\Coloneqq$ & $\one$ & \emph{True}            &\\
%                        & & $\mid$&$ \zero$ & \emph{False}                        &  \\
%                        & & $\mid$&$ f = v$ & \emph{Basic Test}                    &  \\
%                        & & $\mid$&$ a + b$ & \emph{Disjunction}             &  \\
%                        & & $\mid$&$ a \cdot b$ & \emph{Conjunction}         & \textbf{Predicate Semantics: $\pSem{a} \colon \pow\Pk$}\\\cline{6-6}
%                        & & $\mid$&$ \neg a$ & \emph{Negation}               &
%                        % Predicate semantics
%         \multirow{4}{*}{$
%            \begin{array}{lcl}
%            \pSem{\zero} &\triangleq& \emptyset \quad \pSem{\one} \triangleq \Pk\\
%            \pSem{f= v} &\triangleq& \{\pk \mid \pk.f = v,\ \pk \in \Pk\}\\
%            \pSem{a+b} &\triangleq& \pSem{a} \cup \pSem{b} \\
%            \pSem{a\cdot b} \triangleq \pSem{a} \cap \pSem{b}\\
%            \pSem{\neg a} &\triangleq& \Pk \setminus \pSem{a}
%            \end{array}
%            $}\\
%            &&&&&   \\
%         \text{Program}&$\Exp \ni p,q$ & $\Coloneqq$ & $a$ & \emph{Filter}  &   \\
%         \text{expressions} & & $\mid$&$ f \gets n$ & \emph{Modification} & \\
%                         &  & $\mid$&$ p + q$ & \emph{Union} & \\
%                         &  & $\mid$&$ p \cdot q$ & \emph{Sequencing} & \textbf{Guarded String Concatenation}\\\cline{6-6}
%                         &  & $\mid$&$ p^\star$ & \emph{Iteration} &
%                         \multirow{3}{*}{$
%                         \begin{array}{l}
%                         \pk \cdot x \cdot \pk_1 \diamond   \pk_2 \cdot y \cdot \pkp  =
%                         \begin{cases} \pk \cdot x \cdot y  \cdot \pkp & \pk_1 = \pk_2\\
%                         \text{undefined} & \pk_1 \neq \pk_2\end{cases}
%                         \\
%                         X \diamond Y = \{ x\diamond y \mid x\in X, y\in Y \}, \quad X,Y\subseteq \GS
%                         \end{array}
%                         $}\\
%                         &  & $\mid$&$ \text{dup}$ & \emph{Duplication}&  \\
%                         &  & & & & \\[.5em]
%     \end{tabular}
% \caption{\NetKAT: Syntax and Language Model Semantics.}
% \label{fig:synsem}
% \end{figure}


The semantics of \NetKAT is given in terms of \emph{guarded strings},
which are sequences of packets drawn from the set $\GSdef$. Each
guarded string has the form $\pk \cdot \pk_1\cdot \dup \cdots \pk_n
\cdot \dup \cdot \pkp$ where $n\geq 0$. Intuitively, it represents the
packet's history as it goes through the network: $\pk$ is the input
packet, $\pkp$ is the output, the $\pk_i$ are intermediate packets
recorded as it traverses the network. We say two \NetKAT programs
$p_1, p_2 \in \Exp$ are equivalent, written $p_1 \equiv p_2$, whenever
$\Sem{p_1} = \Sem{p_2}$.

\begin{remark}[Atoms]
The reader familiar with prior work on \NetKAT~\cite{netkat} will
notice we define guarded strings slightly differently than in prior
work:
\[
\GSdef \qquad\text{vs.}\qquad \GSdefold.
\]
These representations are isomorphic as $\Pk$ is isomorphic to $\At$,
the atoms of the Boolean algebra \Pred.
\end{remark}

\begin{remark}[Redundancy in the semantics]
We included in \Cref{fig:synsem} an explicit definition of the
semantics $\pSem a$ for $a\in\Pred$, which associates with every
predicate its Boolean interpretation as a set of packets. We present
this explicit definition for notational convenience: $\Pred$ is a
subalgebra of $\Exp$ in which the conjunction and disjunction
operations are compatible with union and sequencing. We could have
equivalently defined $\Sem -$ only on basic tests and negations of
basic tests and then proved that when $\Sem -$ is applied to $a\in
\Pred$, the resulting semantics has the form $\{\pk\cdot\pk \mid \pk
\in \Pk, \pk \in \pSem a\} \in \pow{\Pk\cdot\Pk}$.
\end{remark}


---------------------

A \NetKAT automaton $\mathcal{A}$ is simply a
finite-state NetKAT coalgebra with a distinguished start state $s_0\in
S$, written as $(S, s_0, \epsilon, \delta)$. We will often describe
$\epsilon$ and $\delta$ in terms of families of functions over
$\Pk\times\Pk$. The function $\epsilon\colon S \to 2^{\Pk \times \Pk}$
is in one-to-one correspondence with the family of functions
$\epsilon_{\pk\pkp} \colon S \to 2$, for $\pk,\pkp\in\Pk$ through the
correspondence $ \epsilon_{\pk\pkp} (s) = \epsilon(s)(\pk,\pkp)$;
analogously, the function $\delta\colon S \to S^{\Pk \times \Pk}$ is
in one-to-one correspondence with the family of functions
$\delta_{\pk\pkp} \colon S \to S$ through $\delta_{\pk\pkp}(s) =
\delta (s)(\pk,\pkp)$.

The input to \NetKAT automata are guarded strings of the form $\pk
\cdot \pk_1 \cdot \dup \cdot \ldots \cdot \pk_n \cdot \dup \cdot
\pkp$, for $n\geq 0$. In state $s$, the automaton can either accept
the string if $n=0$ and $\epsilon_{\pk\pkp}(s)= 1$ or it can consume
one packet and $\dup$ from the start of the string and transition to
state $s'$ in which the rest of the string needs to be processed if $n
> 0$. More formally, the set of strings accepted by a state $s\in S$
is captured by the function $\accept\in S \to \pow\GS $ defined
as follows:
    \begin{align}
      \pk \cdot \pkp \in   \accept(s) &
      \iff \epsilon_{\pk\pkp}(s)=1 \\
 \label{accept_transition}
     \pk \cdot \pk_1 \cdot \dup \cdot w \in   \accept(s) &
     \iff
     \pk_1 \cdot w \in \accept(\delta_{\pk\pk_1}(s)).
    \end{align}
The set of strings accepted by an automaton $\mathcal A = (S, s_0,
\epsilon, \delta)$ is defined as:
%
\[
L(\mathcal{A}) \triangleq \accept(s_0).
\]
%
We say that two \NetKAT automata $\mathcal A= (S, s_0, \epsilon^\A,
\delta^\A)$ and $\mathcal B= (T, t_0, \epsilon^\B, \delta^\B)$ are
equivalent if $L(\mathcal{A}) = L(\mathcal{B})$. Our overall goal is
to design efficient procedures for checking equivalence using
bisimulation---a sound and complete proof principle for NetKAT
language equivalence.


\subsection{From \NetKAT Programs to Automata}
\label{automataconversion}

To compute bisimulations between \NetKAT programs, we need to first
convert the programs into automata. We present a direct construction
of a \NetKAT automaton using an analogue of Brzozowski
derivatives~\cite{brzozowski1962}. We first show that the set $\Exp$
has the structure of a $\NetKAT$ coalgebra, that is a \NetKAT
automaton without an initial state and with an infinite state
space. We then prove that for a fixed program $p$ the automaton with
initial state $p$ has a finite state space.

% MM: I think this is irrelevant now?
%% Note that in an implementation, there is an interesting choice to be
%% made here: we can either perform the conversion ``on-the-fly'' along
%% with the bisimulation check, or we can fully convert the programs to
%% automata before starting the bisimulation check. We explore both of
%% these in our implementation and discuss this choice further in
%% \Cref{sec:tradeoffs}.

Given $T\subseteq S$ we say that
$(T,\left<\epsilon\!\!\mid_T,\delta\!\!\mid_T\right>)$ is a
subcoalgebra of $(S,\left<\epsilon,\delta\right>)$ if the restricted
transition function $\delta\!\!\mid_T \colon T \to T^{\Pk\times\Pk}$
is well-defined---i.e., $T$-transitions stay in $T$. We define the
states of the subcoalgebra generated by $s\in S$, denoted
$\left<s\right>$ as the intersection of all subcoalgebras containing
$s$:
\[
\bigcap \{ T \mid (T,\left<\epsilon\vert_T,\delta\vert_T\right>) \text{ is a subcoalgebra of } (S,\left<\epsilon,\delta\right>) \text{ and } s\in T \}
\]
More intuitively, we start from $s$ and close under repeated $\delta$ transitions.

\begin{restatable}[Derivatives for Expressions]{theorem}{automataexpressions}
The set of $\Exp$ has the structure of a \NetKAT coalgebra, given by
the functions $E\colon \Exp \to 2^{\Pk\times\Pk}$ and $D \colon\Exp
\to \Exp^{\Pk\times\Pk}$ in \Cref{fig:derivatives}. Moreover, for
every $p\in\Exp$ the subcoalgebra generated by $p$ is finite as long
as we consider expressions modulo associativity, commutativity, and
idempotence (ACI).
\end{restatable}

With this result, we can now define the \NetKAT automaton
corresponding to an expression.

\begin{definition}[\NetKAT automaton for a program] The \NetKAT automaton $\mathcal A_p$ for a program $p\in\Exp$ is the automaton $(S, s_0, \epsilon, \delta)$, where:
\begin{itemize}
    \item $S = \left<p\right> \subseteq \Exp$ (by taking the subcoalgebra of $(\Exp, \left<E,D\right>)$ modulo ACI).
    \item $s_0 = p$ is the start state.
    \item $\epsilon\colon S \to 2^{\Pk \times\Pk}$ is given by $\epsilon_{\pk\pkp}(s)  = E_{\pk\pkp}(s)$.
        \item $\delta\colon S \to S^{\Pk\times\Pk}$ is given by $\delta_{\pk\pkp}(s) = D_{\pk\pkp}(s)$.
\end{itemize}
\end{definition}

\begin{figure}
\begin{tabular}{|c|c|c|}
    \hline
    $p \in \Exp$        & $E_{\pk} \colon 2^\Pk$            & $D_{\pk\pkp} \colon \Exp\to\Exp$ \\
    \hline
     $a$             & $\{\pk\} \cap [a]$                       & $\zero$  \\
    $f \gets n$         & $\{\pk[n/f]\}$                     & $\zero$ \\
     $\dup$ &  $\emptyset$                                            & $[\pk=\pkp];\pk$\\
    $q + r$             & $E_{\pk}(q) \cup E_{\pk}(r)$     & $D_{\pk\pkp}(q) + D_{\pk\pkp}(r)$\\
    $q \cdot r$         & $\bigcup\limits_{\pkp\in E_\pk(q)}  E_{\pkp}(r)$ & $D_{\pk\pkp}(q); r + \sum\limits_{\gamma} E_{\pk\gamma}(p);D_{\gamma\pkp}(r)$\\
    $q^\star$           & $\{\pk\} \cup \bigcup\limits_{\pkp\in E_\pk(q)}  E_{\pkp}(q^\star)$                    &  $D_{\pk}(q); q^\star + \sum\limits_{\gamma}E_{\pk\gamma}(q) ; D_{\gamma\pkp}(q^\star)$ \\
    \hline
\end{tabular}
    \caption{\NetKAT automata via Brzozowski derivatives. Note that the definitions for iteration are circular, but both are well-defined if we take the least fixpoint of the system of equations. }
    \label{fig:derivatives}
\end{figure}

\begin{example}\label{ex:automaton}
To illustrate the construction, consider the \NetKAT program $p$:
%
\[
p \triangleq
    x=1 +  y=2\cdot y\gets7\cdot\dup\cdot(z=1\cdot y\gets3 +  z=2\cdot z\gets4\cdot\dup\cdot y\gets1)
\]
%
The automaton for $p$ has three states (omitting states with empty
output):
\begin{center}
\begin{tikzpicture}
\node[state, initial] (s0) {$p$};
\node[state, right of=s0] (s1) {$q$};
\node[state, right of=s1] (s2) {$r$};
\node[draw=none, below of=s0, node distance=27pt] {$x=1$};
\node[draw=none, below of=s1, node distance=27pt] {$z=1\cdot y\gets 3$};
\node[draw=none, below of=s2, node distance=27pt] {$y\gets 1$};
\draw (s0) edge node{$y=2\cdot y\gets 7$} (s1);
\draw (s0) edge[ematrix] ++ (0,-0.75);
\draw (s1) edge[ematrix] ++ (0,-0.75);
\draw (s1) edge node{$z=2\cdot z\gets 4$}  (s2);
\draw (s2) edge[ematrix] ++ (0,-0.75);
\end{tikzpicture}
\end{center}

\noindent Here, the states are $p$, $q$ and $r$, where:
%
\[
q \equiv z=1\cdot y\gets3 +  z=2\cdot z\gets4\cdot\dup\cdot y\gets1 \qquad\qquad
r \equiv y\gets 1.
\]
%
To make the drawing above simpler, we have represented both the
transition labels and output labels with dup-free \NetKAT programs.
\end{example}

\subsection{Bisimulations of \NetKAT Automata}

Having reviewed basic definitions for NetKAT, we now present a first
algorithm to check the equivalence of \NetKAT automata $\mathcal{A} =
(S, s_0, \epsilon^\A, \delta^\A)$ and $\mathcal{B} = (T, t_0,
\epsilon^\B, \delta^\B)$. Because NetKAT's acceptance function
$\accept$ is defined on the unique language map $\mathcal L$ of the
Moore automaton corresponding to a \NetKAT automaton, to obtain
language equivalence, our bisimulation relation is not just a relation
over $S\times T$ but $\Pk\times S\times T$ instead.

\begin{definition}[Bisimulation]\label{def:bisim}
    A relation $R
    \subseteq \Pk \times S\times T$ is a \emph{bisimulation} between \NetKAT
    automata $\mathcal{A} = (S, s_0, \epsilon^\A, \delta^\A)$
    and $\mathcal{B} = (T, t_0, \epsilon^\A, \delta^\B)$ if the following conditions hold:
\begin{enumerate}[(a)]
    \item For all $\pk\in\Pk$, $(\pk, s_0, t_0) \in R$.
    \item For all $(\pk,s,t)\in R$:
    \begin{enumerate}[(i)]
    \item $\epsA(s) = \epsB(t)$ \text{ and },
    \item For all $\pkp\in\Pk$,  $(\pkp, \delA(s), \delB(t))\in R$.
    \end{enumerate}
\end{enumerate}
\end{definition}

\noindent \Cref{fig:naive} presents a simple algorithm to compute a
bisimulation between \NetKAT automata.

\begin{restatable}[Correctness of Naive Algorithm]{theorem}{bisimnaivetermination}
The algorithm in \Cref{fig:naive} terminates and either returns true
if a bisimulation exists for the input \NetKAT automata $\mathcal A$
and $\mathcal B$ (and constructs a bisimulation $R$) or false if one
does not exist.
\end{restatable}

\noindent Bisimulations are sound and complete proof principles for
\NetKAT language equivalence.

\begin{figure}
\begin{algorithmic}
\State $R\gets \emptyset; \todoR \gets \{(\pk,s_0,t_0)\mid \pk\in \Pk\}$,
\While{$\todoR$ is not empty}
    \State extract $(\pk, s,t)$ from $\todoR$;
    \If{$(\pk, s,t) \not\in R$}
     \If{$\epsA(s) \neq \epsB(t)$}
      \Return \false;
	\EndIf
      \For {$\pkp \in \Pk$}
     	insert ($\pkp$, $\delA(s)$,$\delB(t)$) in \todoR;
\EndFor
    \EndIf
     insert $(\pk,s,t)$ in $R$;
\EndWhile
\State
\Return \true;
\end{algorithmic}
\caption{Naive NetKAT bisimulation.}
\label{fig:naive}
\end{figure}

\begin{restatable}[Soundness and Completeness]{theorem}{soundcompletenaive}
If there exists a bisimulation $R$ between \NetKAT automata
$\mathcal{A} = (S, s_0, \epsilon^\A, \delta^\A)$ and $\mathcal{B} = (T, t_0,
\epsilon^\B, \delta^\B)$ then $L(\mathcal{A}) = L(\mathcal{B})$.  Conversely, if
$L(\mathcal{A}) = L(\mathcal{B})$ then there exists a bisimulation $R$
between \NetKAT automata $\mathcal{A}$ and $\mathcal B$.
\end{restatable}