This section develops two
algorithms for computing bisimulations that corresopnd to these
paradigms, as well as a hybrid approach that combines the best of both
in a single algorithm. The first is an improvement to the naive
algorithm that takes advantage of the symbolic techniques introduced
in \Cref{sec:symbolic}. We investigate the tradeoffs associated with
using these algorithms in our evaluation.

\paragraph*{Computing Bisimulations Symbolically}\label{sec:fwd-sym}

Using symbolic automata and symbolic packets allows us to collapse
many steps of the concrete bisimulation algorithm into a singlle step
that works symbolically. Recall that a NetKAT bisimulation is a subset
of $\Pk \times S \times T$. We first observe that this type is
equvalent to $S\times T \to \pow \Pk$ via the isomorphism $\pow{\Pk
  \times S \times T} \cong S \times T \to \pow{\Pk}.$ A {\em symbolic
  bisimulation} then, is just a map $S \times T \to \FDD[2]$. Although
our implementation uses this representation of symbolic bisimulations,
to lighten the notation we will often write them as sets of triples
rather than functions.

Next, we show how to reformulate the naive bisimulation algorithm in
terms of symbolic automata and packets. The algorithm is shown in
\Cref{fig:fwd-sym}. Note that we work with the product automaton for
the input automata, $\mathcal{A}$ and $\mathcal{B}$. The definition is
standard and given in \Cref{app:product}. Also, we not only replace
packets with symbolic packets, but we also compute a symbolic
bisimulation.

\begin{figure} \small
\begin{subfigure}[t]{.49\linewidth}
\begin{algorithmic}
\State $R\gets \emptyset$;
\State $\todoR \gets \{(\Pk,s_0,t_0)\}$;
\While{$\todoR$ is not empty}
    \State extract $(\sympk, s,t)$ from $\todoR$;
    \If {$\sympk \not\subseteq R_{s,t}$}
     \If {$\epsilon_{\sympk}^{\A}(s) \neq \epsilon_{\sympk}^{\B}(t)$}
        \State \Return \false;
	\EndIf
      \For {$(\sympkp, (s', t')) \in \nextC\ (s,t)\ (\sympk)$}
        \If {$\sympkp \neq \emptyset$}
     	    \State insert ($\sympkp, s', t'$) in \todoR;
        \EndIf
      \EndFor
     \State $R_{s,t} \gets R_{s,t} \cup \sympk$;
    \EndIf
\EndWhile
\State
\Return \true;
\end{algorithmic}
\caption{Forward}
\label{fig:fwd-sym}

\end{subfigure}
\begin{subfigure}[t]{.49\linewidth}
\begin{algorithmic}
\State $B \gets \emptyset$;
  \State $\todoR \gets \{ (\{\pk \mid \epsA(s) \neq \epsB(t)\}, s, t) \mid \sympk \neq \emptyset\}$;
\While{$\todoR$ is not empty}
    \State extract $(\sympkp, s', t')$ from $\todoR$;
    \If {$\sympkp \nsubseteq B_{s,t}$}
        \If {$s' = s_0 \wedge t' = t_0$}
            \State \Return \false;
        \EndIf
        \For {$(\sympk, (s, t)) \in \prevC\ (s',t')\ (\sympkp)$}
          \If {$\sympk \neq \emptyset$}
            \State insert ($\sympk, s, t$) in \todoR;
          \EndIf
        \EndFor
        \State $B_{s,t} \gets B_{s,t} \cup \sympkp$;
    \EndIf
\EndWhile
\Return \true
\end{algorithmic}
\caption{Backward}\label{fig:bkwd-alg}
\end{subfigure}

\caption {Two symbolic algorithms for deciding equivalence of \NetKAT
automata.}\label{fig:pseudocode}
\end{figure}

\paragraph*{Computing Bisimulations Backwards}

Next, we show that the opposite approach (similar to Moore's algorithm
for minimization) also makes sense in the \NetKAT
setting. Conceptually, while the forward method finds triples in
$\PkST$ that must be in every bisimulation, the backwards method will
search for for triples in $\PkST$ that can never be in a
bisimulation. In particular, we start by relating states that have
\emph{different} output functions, and follow transitions
\emph{backward} along transitions in the two automata (equivalently,
the product automaton). Ultimately, if we find that the start states
can be related under this scheme, then the two automata are
inequivalent (since we have found a path leading to different behavior
in the automata). On the other hand, if we exhaust backward steps and
have still not related the start states, then we conclude the automata
are equivalent (see \Cref{thm:bkwd-correct}).



\subsection{Correctness of the backward algorithm}\label{sec:bkwd-correct}

We are ready to show the correctness of \Cref{fig:bkwd-alg} as a
decision procedure for the equivalence of two deterministic \NetKAT
automata.

\begin{lemma}\label{lem:bkwd-termination}
The algorithm in \Cref{fig:bkwd-alg} terminates.
\end{lemma}
\begin{proof}
Let $n = |\Pk\times S\times T - B|$ and let $m = |\todoR|$.  The tuple
$(n,m)$ is always finite and is bounded below by $(n,0)$, for any $n$,
because of the while loop condition.  Moreover $(n,m)$ must decrease
lexicographically every loop iteration: on iterations where we add to
$B$, then $n$ decreases, and on iterations where we do not add to $B$,
then $n$ is constant and $m$ decreases from popping $\todoR$. Note
that we cannot add to $\todoR$ on iterations where $B$ does not
decrease, and that $\todoR$ only contains triples with nonempty
symbolic packets, because this is checked in both places triples are
added to $\todoR$.
\end{proof}

\begin{restatable}{theorem}{bkwdcorrect}\label{thm:bkwd-correct}
The algorithm in \Cref{fig:bkwd-alg} returns \true if there is a
bisimulation between $\mathcal{A}$ and $\mathcal{B}$ and returns \false if
there is no such bisimulation.
\end{restatable}