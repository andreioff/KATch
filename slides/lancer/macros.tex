\usepackage[utf8]{inputenc}
\usepackage[T1]{fontenc}
% \usepackage{amssymb}
\usepackage{xspace}
% \usepackage[bitstream-charter]{mathdesign}
\usepackage{microtype}
\usepackage{textcomp}
\usepackage{mathtools}
\usepackage{hyperref}
% \usepackage[all]{hypcap}
\hypersetup{
    colorlinks=true,
    linkcolor=blue,
    filecolor=magenta,
    urlcolor=blue,
}
\usepackage[nameinlink,noabbrev]{cleveref}
\usepackage{float}
\usepackage{listings}
% \usepackage{color}
% \usepackage[dvipsnames]{xcolor}
% \PassOptionsToPackage{usenames,dvipsnames}{xcolor}
\usepackage{tcolorbox}
\usepackage{stmaryrd}
\usepackage{slashed}

\Crefname{section}{\S\!}{\S}

\definecolor{lightgray}{rgb}{.9,.9,.9}
\definecolor{darkgray}{rgb}{.4,.4,.4}
\definecolor{purple}{rgb}{0.65, 0.12, 0.82}
\definecolor{ultramarine}{rgb}{0.17, 0.26, 0.70}

\newcommand{\lineref}[1]{\hyperref[#1]{Line~\ref*{#1}}}

\lstdefinelanguage{JavaScript}{
  keywords=[1]{c1,c1',c2,c2',c},
  keywords=[2]{fork, send, receive, close},
  keywordstyle=[1]\color{ultramarine},
  keywordstyle=[2]\color{BrickRed},
  comment=[l]{//},
  morecomment=[s]{/*}{*/},
  commentstyle=\color{purple}\ttfamily,
  stringstyle=\color{red}\ttfamily,
  columns=fullflexible
}

\lstset{
   language=JavaScript,
   extendedchars=true,
   basicstyle=\small\ttfamily,
   morekeywords=[2]{let,send,recv,fork},
   columns=flexible,
   mathescape=true,
   literate=
    {forall}{$\forall$}1
    {->}{$\to$}1
    {=>}{$\implies$}1
    {fun}{$\lambda$}3
    {...}{$\dots$}3
}

\newcommand{\N}{\mathbb{N}}
\newcommand{\Z}{\mathbb{Z}}
\newcommand{\Q}{\mathbb{Q}}
\newcommand{\R}{\mathbb{R}}

\usepackage{amsmath}
\usepackage{amsthm}
\usepackage{mathpartir}

% \newtheorem{theorem}{Theorem}[section]
% \newtheorem{lemma}[theorem]{Lemma}
% \theoremstyle{definition}
% \newtheorem{definition}{Definition}[section]
% \newtheorem{corollary}{Corollary}[theorem]
\usepackage{adjustbox}
% \usepackage[notref,notcite]{showkeys}
\usepackage{todonotes}

\usepackage{xparse}

\NewDocumentCommand{\TODO}{m m}%
  {{\bfseries\color{#1}#2}}%
\newcommand*{\robbert}[1]{{\TODO{red}{RK: #1}}}
\newcommand*{\jules}[1]{{\TODO{teal}{JJ: #1}}}
\newcommand*{\stephanie}[1]{\TODO{blue}{SB: #1}}
%Comment out to disable comments
\RenewDocumentCommand{\TODO}{m m}{}

% =========== TIKZ ===========
\usepackage{graphicx}
\usepackage{tikz}
\usetikzlibrary{shapes.geometric, arrows}
\usetikzlibrary{fit}
\usetikzlibrary{svg.path}
% \usetikzlibrary{graphdrawing}
% \usetikzlibrary{graphdrawing.force}
% \usetikzlibrary{graphdrawing.layered}
\usetikzlibrary{decorations}
\usetikzlibrary{decorations.markings}
\usetikzlibrary{backgrounds}
\usetikzlibrary{tikzmark}

\tikzstyle{thread} = [thick,rectangle, rounded corners, minimum width=0.6cm, minimum height=0.6cm,text centered, draw=black, fill=red!45, scale=0.9]
\tikzstyle{threadg} = [ultra thick, dotted, rectangle, rounded corners, minimum width=0.6cm, minimum height=0.6cm,text centered, draw=black, fill=LimeGreen!70, scale=0.9]
\tikzstyle{threadt} = [thick,rectangle, rounded corners, minimum width=0.6cm, minimum height=0.6cm,text centered, draw=black, scale=0.9, fill=black!30, opacity=0.5]
\tikzstyle{threadu} = [thick,rectangle, rounded corners, minimum width=0.6cm, minimum height=0.6cm,text centered, draw=black, scale=0.9]

\tikzstyle{dummy} = [inner sep=0, minimum width=0.63cm, scale=0.9]
\tikzstyle{channel} = [thick,circle, text centered, draw=black, fill=blue!30, inner sep=0, minimum width=0.63cm, scale=0.9]
\tikzstyle{unknown} = [thick,circle, text centered, draw=black, fill=white!30, scale=0.9, inner sep=0, minimum width=0.63cm]
\tikzstyle{val} = [thick,rectangle, rounded corners, minimum width=1cm, minimum height=0.8cm, text centered, draw=black, dashed, fill=green!30, scale=0.9]
\tikzstyle{lock} = [thick,circle, text centered, draw=black, fill=green!30, scale=0.9]
\tikzstyle{arrow} = [thick,->,>=stealth]
\newcommand{\triplearrow}[2]{
  \draw [arrow, bend right] (#1) edge (#2);
  \draw [arrow, bend left] (#1) edge (#2);
  \draw [arrow] (#1) -- (#2);
}
\tikzset{->-/.style={thick,decoration={
    markings,
    mark=at position 0.68 with {\arrow[black,scale=1.7]{latex}},
    mark=at position 0.6 with {\arrow[red,scale=1.3]{latex}},
    mark=at position 1 with {\arrow{stealth}}},
  postaction={decorate}}}

\tikzset{-<-/.style={thick,decoration={
    markings,
    mark=at position 0.6 with {\arrow[black,scale=1.7]{latex reversed}},
    mark=at position 0.6 with {\arrow[red,scale=1.3]{latex reversed}},
    mark=at position 1 with {\arrow{stealth}}},
  postaction={decorate}}}

\tikzset{-->/.style={thick,decoration={
    markings,
    mark=at position 1 with {\arrow{stealth}}},
  postaction={decorate}}}

\newif\ifshowtikz
\showtikztrue
% \showtikzfalse   % <---- comment/uncomment that line

\let\oldtikzpicture\tikzpicture
\let\oldendtikzpicture\endtikzpicture

\renewenvironment{tikzpicture}{%
    \ifshowtikz\expandafter\oldtikzpicture%
    \else\comment%
    \fi
}{%
    \ifshowtikz\oldendtikzpicture%
    \else\endcomment%
    \fi
}

% =============================

\newcommand{\ie}{\textit{i.e.,}\xspace}
\newcommand{\eg}{\textit{e.g.,}\xspace}
\newcommand{\etc}{\textit{etc.}\xspace}
\newcommand{\viceversa}{\textit{vice versa}}
\newcommand{\wrt}{w.r.t.\ }

\newcommand{\Type}{\kern-0.1em\textdom{Type}}
\newcommand{\Session}{\textdom{Session}}
\newcommand{\Heap}{\textdom{Heap}}
\newcommand{\Cfg}{\textdom{Cfg}}
\newcommand{\Chan}{\textdom{Chan}}

\newcommand{\letkeyword}{\mathsf{let}}
\newcommand{\sendkeyword}{\mathsf{send}}
\newcommand{\recvkeyword}{\mathsf{receive}}
\newcommand{\forkkeyword}{\mathsf{fork}}
\newcommand{\closekeyword}{\mathsf{close}}

\newcommand{\app}[2]{#1\ #2}
\newcommand{\lam}[2]{\lambda #1.\ #2}
\newcommand{\send}[2]{\sendkeyword(#1,#2)}
\newcommand{\recv}[1]{\recvkeyword(#1)}
\newcommand{\letV}[3]{\letkeyword\ #1 = #2\ \mathsf{in}\ #3}
\newcommand{\letU}[2]{\letkeyword\ () = #1\ \mathsf{in}\ #2}
\newcommand{\letP}[4]{\letkeyword\ (#1,#2) = #3\ \mathsf{in}\ #4}
\newcommand{\ifE}[3]{\mathsf{if}\ #1\ \mathsf{then}\ #2\ \mathsf{else}\ #3}
\newcommand{\fork}[1]{\forkkeyword(#1)}
\newcommand{\close}[1]{\closekeyword(#1)}

\newcommand{\typed}[3]{#1 \vdash #2 : #3}
\newcommand{\typedH}[4]{#1; #2 \vdash #3 : #4}
\newcommand{\typedSL}[3]{#1 \vDash #2 : #3}
\newcommand{\buftyped}[3]{{}\vdash_{\mathit{buf}}{} #1 : (#2,#3)}

\newcommand{\heapEnv}{\Sigma}
\newcommand{\nattype}{\mathbf{N}}
\newcommand{\unittype}{\mathbf{1}}
\newcommand{\voidtype}{\mathbf{0}}
\newcommand{\funtype}{\mathrel{-\kern-1pt\circ}}

\renewcommand{\Loc}{\textdom{Addr}}

\newcommand{\loc}{a}
\newcommand{\chan}{c}
\newcommand{\session}{s}
\newcommand{\heap}{h}
\renewcommand{\cfg}{\rho}
\newcommand{\buf}{\vec{v}}
\newcommand{\ctx}{\Gamma}

\newcommand{\ctxdisjoint}{\mathrel{\bot}}

\newcommand{\hole}{\Box}
\newcommand{\nexteq}{\\[5pt]}
\newcommand{\chanep}[2]{\#(#1,#2)}
\newcommand{\thread}[1]{#1}

\renewcommand{\step}{\leadsto}
\newcommand{\steps}{\step^*}
\newcommand{\pstep}{\leadsto_\mathrm{pure}}
\newcommand{\hstep}{\leadsto_\mathrm{head}}
\newcommand{\gstep}{\leadsto_\mathrm{global}}
\newcommand{\append}{+\kern-1ex+\kern0.8ex}
\newcommand{\length}[1]{| #1 |}
\newcommand{\blockedShort}[1]{\mathsf{blocked}_{#1}}
\newcommand{\blocked}[3][]{\blockedShort{#1}(#2, #3)}

\newcommand{\RECV}{\textnormal{\textbf ?}}
\newcommand{\SEND}{\textnormal{\textbf !}}
\newcommand{\sRecvHead}[1]{\RECV\, #1}
\newcommand{\sSendHead}[1]{\SEND\, #1}

\newcommand{\sRecv}[2]{\sRecvHead{#1}.\, #2}
\newcommand{\sSend}[2]{\sSendHead{#1}.\, #2}
\newcommand{\sEnd}{\mathsf{End}}
\newcommand{\dual}[1]{\overline{#1}}

\newcommand{\mset}{\Delta}

\newcommand{\sep}{\ast}
\newcommand{\pure}[1]{\ulcorner #1 \urcorner}
\newcommand{\own}[1]{\mathsf{own}(#1)}
\newcommand{\sepand}{\ \, \ast\ \,}

\newcommand{\cgraph}{G}
\newcommand{\CgraphShort}{\textdom{Cgraph}}
\newcommand{\Cgraph}[2]{\CgraphShort(#1,#2)}

\newcommand{\graph}{G}
\newcommand{\graphB}{H}
\newcommand{\GraphShort}{\textdom{graph}}
\newcommand{\Graph}[2]{\GraphShort(#1,#2)}

\newcommand{\Vertex}{\textdom{V}}
\newcommand{\Lab}{\textdom{L}}

\newcommand{\vertex}{\nu}
\newcommand{\vertexB}{\mu}
\newcommand{\vertexC}{\eta}
\newcommand{\lab}{l}
\newcommand{\labB}{l'}
\newcommand{\tvar}{\alpha}
\newcommand{\tvarB}{\alpha}
\newcommand{\chtag}{t}

\newcommand{\edge}[4][]{#2 \to^{#3}_{#1} #4}
\newcommand{\edgeReverse}[4][]{#2 \leftarrow^{#3}_{#1} #4}
\newcommand{\connectedOne}[3][]{#2 \leftrightarrow_{#1} #3}
\newcommand{\connected}[3][]{#2 \leftrightarrow^*_{#1} #3}

\newcommand{\notconnected}[3][]{#2 \mathrel{\slashed{\leftrightarrow}_{#1}^*} #3}

\newcommand{\inlabels}[2]{\mathsf{in}(#1,#2)}
\newcommand{\outedges}[2]{\mathsf{out}(#1,#2)}
\newcommand{\multiset}{\Delta}
\newcommand{\Multiset}[1]{\textdom{Multiset}\;#1}
\newcommand{\typedef}{:}
\newcommand{\ThreadV}{\mathsf{Thread}}
\newcommand{\ChanV}{\mathsf{Chan}}

\newcommand{\wflocal}[1][]{\mathsf{wf}^{\mathit{local}}_{#1}}

\newcommand{\wf}{\mathsf{wf}}
\newcommand{\init}{\mathsf{init}}
\newcommand{\final}{\mathsf{final}}

\newcommand{\wandrules}{\mprset{fraction={{\text{-}} {\text{-}} {\text{ }\ast}}}}
\newcommand{\genwfShort}{\mathsf{wf}}
\newcommand{\genwf}[1]{\mathsf{wf}(#1)}
\newcommand{\linv}{P}
\newcommand{\linvB}{P'}
\newcommand{\msetA}{\multiset_1}
\newcommand{\msetB}{\multiset_2}
\newcommand{\activeset}{\mathsf{active}}
\newcommand{\canstep}[1]{#1 \text{ can step}}